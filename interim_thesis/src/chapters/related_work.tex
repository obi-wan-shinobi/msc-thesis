\chapter{\label{cha:related}Related Work}

\section{Neural Tangent Kernel}\label{sec:ntk}

As discussed in Chapter~\ref{cha:intro}, for nonlinear neural networks the
gradient flow dynamics are parameter-dependent and do not admit a simple
closed-form description. One approach to recovering a tractable analytical
framework is to consider regimes in which the network behaves approximately
linearly around its random initialization. The \emph{Neural Tangent Kernel}
(NTK), introduced by \citet{jacot2018ntk}, formalizes this idea by studying
training dynamics through a first-order linearization in parameter space,
leading to a kernel-based description of learning in wide neural networks.

\subsection{Linearization around initialization}

Let $f(x;\theta)$ denote a neural network with parameters
$\theta \in \mathbb{R}^p$, initialized at $\theta_0$. A first-order Taylor
expansion around $\theta_0$ yields
\begin{equation}
	f(x;\theta)
	\;\approx\;
	f(x;\theta_0)
	+
	\nabla_\theta f(x;\theta_0)^\top (\theta - \theta_0),
	\label{eq:ntk-linearization}
\end{equation}
where $f(x;\theta_0)$ is the network output at initialization and
\[
	\phi(x) := \nabla_\theta f(x;\theta_0)
\]
defines the tangent feature map. Locally around initialization, the network
thus behaves as a linear model in parameter space,
\begin{equation}
	f(x;\theta) \approx f(x;\theta_0) + \phi(x)^\top (\theta - \theta_0).
	\label{eq:ntk-linear-model}
\end{equation}

\begin{definition}[Neural Tangent Kernel \citep{jacot2018ntk}]
	Given an initialization $\theta_0$, the neural tangent kernel is defined as
	\[
		\Theta_0(x,x')
		:= \nabla_\theta f(x;\theta_0)^\top
		\nabla_\theta f(x';\theta_0).
	\]
\end{definition}

The NTK measures how infinitesimal parameter updates couple the network outputs
at different inputs and plays the role of an empirical covariance operator in
the tangent feature space.

\subsection{Training dynamics induced by the NTK}

We consider training with the squared loss
\[
	L(\theta) = \frac{1}{2}\sum_{i=1}^n \bigl(f(x_i;\theta) - y_i\bigr)^2
\]
using gradient descent with step size $\eta>0$,
\[
	\theta_{t+1} = \theta_t - \eta \nabla_\theta L(\theta_t),
	\qquad t = 0,1,2,\dots
\]
Let $f_t(x_i) := f(x_i;\theta_t)$ the gradient of the loss is
\[
	\nabla_\theta L(\theta_t)
	=
	\sum_{j=1}^n \bigl(f_t(x_j)-y_j\bigr)\,\nabla_\theta f(x_j;\theta_t).
\]
Applying a first-order Taylor expansion of $f(\cdot;\theta)$ around $\theta_t$
gives
\[
	f_{t+1}(x_i)
	\approx
	f_t(x_i)
	+
	\nabla_\theta f(x_i;\theta_t)^\top(\theta_{t+1}-\theta_t).
\]
Substituting the gradient descent update
$\theta_{t+1}-\theta_t = -\eta \nabla_\theta L(\theta_t)$
yields
\[
	f_{t+1}(x_i)
	=
	f_t(x_i)
	-
	\eta
	\sum_{j=1}^n
	\nabla_\theta f(x_i;\theta_t)^\top
	\nabla_\theta f(x_j;\theta_t)\,
	\bigl(f_t(x_j)-y_j\bigr),
\]
which can be written compactly as
\begin{equation}
	f_{t+1}(x_i)
	=
	f_t(x_i)
	-
	\eta
	\sum_{j=1}^n
	\Theta_t(x_i,x_j)\bigl(f_t(x_j)-y_j\bigr),
	\label{eq:ntk-prediction-dynamics}
\end{equation}
where the iteration-dependent NTK is
\[
	\Theta_t(x_i,x_j)
	:=
	\nabla_\theta f(x_i;\theta_t)^\top
	\nabla_\theta f(x_j;\theta_t).
\]
If we stack predictions into
$f_t := (f_t(x_1),\dots,f_t(x_n))^T \in \mathbb{R}^n$, in vector form, the dynamics read
\begin{equation}
	f_{t+1} = f_t - \eta\,\Theta_t (f_t - y).
	\label{eq:ntk-vector-dynamics}
\end{equation}
where $
	\bar\Theta
	=
	\bigl(\bar\Theta(x_i,x_j)\bigr)_{i,j=1}^n
	\in \mathbb{R}^{n\times n}$ is the kernel gram matrix.
In general, $\Theta_t$ evolves during training, reflecting changes in the
network’s tangent features.

\subsection{One-hidden-layer NTK and infinite-width limit}

To make the NTK explicit, consider a two-layer network of width $m$,
\begin{equation}
	f(x)
	=
	\frac{1}{\sqrt{m}}
	\sum_{\alpha=1}^m
	v_\alpha\,\varphi(w_\alpha^\top x),
	\qquad
	v_\alpha \in \mathbb{R}, \quad w_\alpha, x \in \mathbb{R}^d.
	\label{eq:1hl-network}
\end{equation}
Each neuron is parameterized by $\theta_\alpha = (v_\alpha,w_\alpha)$ and
initialized as
\[
	v_\alpha \sim \mathcal{N}(0,\sigma_v^2),
	\qquad
	w_\alpha \sim \mathcal{N}\!\left(0,\frac{\sigma_w^2}{d}I_d\right),
\]
independently across $\alpha = 1,\dots,m$. $\varphi: \mathbb{R} \to \mathbb{R}$ is a non-linear activation function.

\paragraph{Finite-width NTK at initialization.}
By direct computation,
\begin{align}
	\nabla_{v_\alpha} f(x)
	 & = \frac{1}{\sqrt{m}}\varphi(w_\alpha^\top x),                 \\
	\nabla_{w_\alpha} f(x)
	 & = \frac{1}{\sqrt{m}}\,v_\alpha\,\varphi'(w_\alpha^\top x)\,x.
\end{align}
Substituting into the definition of the NTK yields the empirical kernel
\begin{equation}
	\Theta_0^{(m)}(x,x')
	=
	\frac{1}{m}\sum_{\alpha=1}^m
	\varphi(w_\alpha^\top x)\,\varphi(w_\alpha^\top x')
	+
	\frac{1}{m}\sum_{\alpha=1}^m
	v_\alpha^2\,
	\varphi'(w_\alpha^\top x)\,
	\varphi'(w_\alpha^\top x')\,
	x^\top x'.
	\label{eq:ntk-1hl-finite}
\end{equation}

\paragraph{Infinite-width limit.}
Each sum in \eqref{eq:ntk-1hl-finite} is an empirical average of i.i.d.\ terms.
By the strong law of large numbers,
\[
	\Theta_0^{(m)}(x,x')
	\;\xrightarrow{\text{a.s.}}\;
	\bar\Theta(x,x')
	\qquad \text{as } m \to \infty,
\]
where the limiting NTK is deterministic and given by
\begin{equation}
	\bar\Theta(x,x')
	=
	\mathbb{E}_{w}\!\big[\varphi(w^\top x)\,\varphi(w^\top x')\big]
	+
	\sigma_v^2\,x^\top x'\,
	\mathbb{E}_{w}\!\big[\varphi'(w^\top x)\,\varphi'(w^\top x')\big].
	\label{eq:ntk-1hl-infinite}
\end{equation}
This argument extends to deep fully connected networks, where both the forward
covariance kernel and the NTK satisfy recursive layerwise equations in the
infinite-width limit \citep{jacot2018ntk, lee2020wide}.

\subsection{Constant-kernel regime and spectral dynamics}

In the infinite-width limit, the NTK remains constant throughout training,
$\Theta_t \equiv \bar\Theta$. Equation~\eqref{eq:ntk-vector-dynamics} then reduces
to the linear iteration
\begin{equation}
	f_{t+1} = f_t - \eta\,\bar\Theta (f_t - y).
	\label{eq:ntk-constant-kernel}
\end{equation}
Letting $r_t := f_t - y$, we obtain
\[
	r_{t+1} = (I - \eta \bar\Theta)\,r_t.
\]
Assuming $\eta < 2/\lambda_{\max}(\bar\Theta)$, the residual admits the closed
form
\begin{equation}
	r_t = (I - \eta \bar\Theta)^t r_0,
	\qquad
	f_t = y + (I - \eta \bar\Theta)^t (f_0 - y).
	\label{eq:ntk-solution}
\end{equation}
As in the linear setting of Chapter~\ref{cha:intro}, convergence occurs
independently along the eigenvectors of $\bar\Theta$, with geometric rates
determined by the corresponding eigenvalues.

\subsection{NTK dynamics, solution structure, and implicit bias}

In the constant-kernel (infinite-width) regime, the NTK prediction dynamics
reduce to a linear iteration in prediction space. Eq \ref{eq:ntk-constant-kernel},
admits fixed points $f^\star$ satisfying
\begin{equation}
	\bar\Theta\,(f^\star - y) = 0.
	\label{eq:ntk-fixed-point}
\end{equation}
Equation~\eqref{eq:ntk-fixed-point} characterizes the set of global minimizers of
the squared loss in the NTK regime. As in linear least squares, the structure of
this solution set depends on the rank of the operator $\bar\Theta$.

If $\bar\Theta$ is strictly positive definite on the training set, the solution
is unique and satisfies $f^\star = y$. When $\bar\Theta$ is singular, the loss
admits infinitely many interpolating solutions in prediction space, differing
by elements of $\ker(\bar\Theta)$. In this case, gradient descent converges to a
particular fixed point determined by the initialization.

\paragraph{Tangent features and operator factorization.}
To formulate the NTK in a way that remains meaningful in the infinite-width
limit, it is convenient to view parameter perturbations as elements of a
Hilbert space $\mathcal{H}_\theta$ equipped with an inner product
$\langle\cdot,\cdot\rangle$. In the finite-dimensional case,
$\mathcal{H}_\theta=\mathbb{R}^p$ with the Euclidean inner product; in the
infinite-width limit, $\mathcal{H}_\theta$ denotes the corresponding limit
space of parameter perturbations, often referred to as the tangent parameter
space \citep{jacot2018ntk, lee2020wide}.

In this setting, the limiting NTK Gram matrix on the training set can be
expressed as
\begin{equation}
	\bar\Theta = J_0 J_0^\ast,
\end{equation}
where $J_0 : \mathcal{H}_\theta \to \mathbb{R}^n$ denotes the Jacobian of the
network outputs with respect to parameters at initialization, viewed as a
linear operator,
\[
	(J_0 h)_i = \langle \nabla_\theta f(x_i;\theta_0),\,h\rangle,
\]
and $J_0^\ast$ is its adjoint. This representation makes explicit that
$\bar\Theta$ is symmetric and positive semidefinite, and that it acts as an
empirical covariance operator for the tangent features induced by the network
at initialization \citep{jacot2018ntk}.

\paragraph{Implicit bias and representer viewpoint.}
When $\bar\Theta$ is singular, the Moore--Penrose pseudoinverse $\bar\Theta^+$
acts as the inverse on $\mathrm{range}(\bar\Theta)$ and annihilates
$\ker(\bar\Theta)$. Consequently, constant-kernel NTK training eliminates only
the residual component lying in $\mathrm{range}(\bar\Theta)$ while preserving
the component in $\ker(\bar\Theta)$. This behavior reflects an implicit
regularization effect analogous to linear least squares and can be interpreted
as convergence to a minimum-norm solution in the reproducing kernel space
associated with $\bar\Theta$, as formalized by representer-theorem results
\citep{bietti2019inductivebiasneuraltangent,
	bartolucci2021understandingneuralnetworksreproducing}.

\paragraph{Limitations of stable kernels and feature learning.}
Although the NTK framework provides a clean and tractable description of
training dynamics, the stability of the kernel also introduces inherent
limitations. When the kernel does not change during training, learning is
effectively confined to a fixed feature space. Indeed, any positive
semidefinite kernel admits a representation
$K(x,x')=\langle \Phi(x),\Phi(x')\rangle_{\mathcal H}$, so that optimization in
the NTK regime corresponds to fitting a linear model in the associated
reproducing kernel Hilbert space $\mathcal H$.

In contrast, for finite-width networks the NTK typically evolves during
training, implying that the tangent features
$\Phi_t(x)=\nabla_\theta f(x;\theta_t)$ also change over time. This evolution
allows the network to adapt its representation to the data, a phenomenon
commonly referred to as \emph{feature learning}. Such effects are absent in a
strictly stable-kernel regime, where only the coefficients of fixed features
are adjusted.

The importance of feature learning is already visible in simplified settings.
Even purely linear networks exhibit nontrivial training dynamics due to the
coupling of parameters across layers, leading to implicit biases that cannot
be captured by a fixed kernel model
\citep{saxe2014exactsolutionsnonlineardynamics, tu2024mixeddynamicslinearnetworks}.
From this viewpoint, the NTK regime can be seen as a useful but restrictive
approximation, which motivates studying regimes where kernel evolution and
feature learning play an explicit role.

\section{Mean-field representation of a two-layer network}
While the neural tangent kernel provides a linearized description of training
dynamics around initialization, it does not capture regimes in which features
evolve significantly during optimization. An alternative and complementary
approach is provided by the \emph{mean-field} perspective, which studies wide
neural networks by viewing their parameters as interacting particles and
describing training as the evolution of a probability distribution over
parameter space.

As the network width tends to infinity under appropriate scalings, the discrete
training dynamics induced by stochastic gradient descent converge to a
deterministic evolution equation for the parameter distribution. This viewpoint
was first formalized for two-layer networks by \citet{Mei2018meanfieldview}, who derived a
nonlinear partial differential equation governing the evolution of the parameter
measure. Related interacting particle system formulations and convergence
results were developed by \citet{Rotskoff_2022}, while extensions to multilayer
networks were proposed by \citet{nguyen2019meanfieldlimitlearning}. From an
optimization perspective, \citet{chizat2018globalconvergencegradientdescent} showed that, under
suitable initialization and in the many-particle limit, training dynamics can
be interpreted as a Wasserstein gradient flow on the space of probability
measures, and that this flow converges to global minimizers despite the
non-convexity of the finite-dimensional parameterization.

In the following, we adopt this mean-field viewpoint and reformulate a two-layer
network as a linear functional of an empirical measure on parameter space, which
serves as the starting point for distributional descriptions of training
dynamics.

Consider a two-layer neural network of the form
\begin{equation}
	f_\theta(x)
	=
	\frac{1}{m}
	\sum_{\alpha=1}^m
	v_\alpha\,\varphi(w_\alpha^\top x),
	\qquad
	v_\alpha \in \mathbb{R}, \;
	w_\alpha, x \in \mathbb{R}^d, \;
	\label{eq:mf-finite-network}
\end{equation}
where each neuron is parameterized by
\(
\theta_\alpha = (v_\alpha, w_\alpha)
\), $\alpha \in \{1,\ldots,m\}$

\subsection{Parameter space and empirical measure.}
Let $\Omega := \mathbb{R} \times \mathbb{R}^d$
denote the parameter space, and let $\mathcal{F}$ be its Borel $\sigma$-algebra.
For each neuron $\alpha$, define the Dirac measure $\delta_{\theta_\alpha}$ on
$(\Omega,\mathcal{F})$, i.e. if $A \in \mathcal{F}$ then,
\[
	\delta_{\theta_\alpha}(A)
	=
	\begin{cases}
		1, & \theta_\alpha \in A, \\
		0, & \text{otherwise}.
	\end{cases}
\]
Define the empirical neuronal measure
\begin{equation}
	\mu_m := \frac{1}{m}\sum_{\alpha=1}^m \delta_{\theta_\alpha}.
	\label{eq:empirical-measure}
\end{equation}
By construction, $\mu_m \ge 0$ and  $\mu_m(\Omega)=1$, hence $\mu_m \in
	\mathcal P(\Omega)$ where $\mathcal P(\Omega)$ is the space of Borel
probability measures on $\Omega$.

\paragraph{Integration against Dirac measures.}
Let $\chi_A$ denote the indicator function of a measurable set $A \subset \Omega$.
Then, for any $\theta_\alpha \in \Omega$,
\[
	\int_\Omega \chi_A(\theta)\,d\delta_{\theta_\alpha}(\theta)
	=
	\delta_{\theta_\alpha}(A)
	=
	\chi_A(\theta_\alpha).
\]
More generally, if $s(\theta) = \sum_{k=1}^r c_k \chi_{A_k}(\theta)$ is a simple
function, then
\[
	\int_\Omega s(\theta)\,d\delta_{\theta_\alpha}(\theta)
	=
	\sum_{k=1}^r c_k \chi_{A_k}(\theta_\alpha)
	=
	s(\theta_\alpha).
\]
Any nonnegative measurable function $g : \Omega \to \mathbb{R}$ can be approximated by
a sequence of simple functions,$\{s_n\}_{n\ge 0}$. If $s_n \uparrow g$ and each $s_n \ge 0$ then
by Monotone Convergence Theorem,
\begin{equation}
	\int_\Omega g\,d\delta_{\theta_\alpha}
	= \int_\Omega \lim_{n \to \infty} s_n\,d\delta_{\theta_\alpha}
	=  \lim_{n \to \infty} \int_\Omega s_n\,d\delta_{\theta_\alpha}
	=  \lim_{n \to \infty} s_n(\theta_\alpha)
	= g(\theta_\alpha).
	\label{eq:dirac-integration}
\end{equation}

\paragraph{Integral representation of the network.}
Consider now the integral of $g$ with respect to the empirical measure $\mu_m$:
\[
	\int_\Omega g\,d\mu_m
	=
	\int_\Omega g\,d\left(\frac{1}{m}\sum_{\alpha=1}^m \delta_{\theta_\alpha}\right)
	=
	\frac{1}{m}\sum_{\alpha=1}^m g(\theta_\alpha).
\]
Define
\[
	g(\theta_\alpha) = g(v_\alpha,w_\alpha) := v_\alpha\,\varphi(w_\alpha^\top x).
\]
Then
\begin{equation}
	\int_\Omega g\,d\mu_m
	=
	\frac{1}{m}\sum_{\alpha=1}^m v_\alpha\,\varphi(w_\alpha^\top x)
	=
	f_\theta(x),
	\label{eq:mf-integral-form}
\end{equation}
recovering the finite-width network \eqref{eq:mf-finite-network}.

\paragraph{Mean-field viewpoint.}
Equation~\eqref{eq:mf-integral-form} shows that the network output can be written
as a linear functional of the empirical measure $\mu_m$.
This representation is only possible because we scale the output by $1/m$ and
because the hidden neurons are permutation
invariant. This implies that the model depends on the
parameters $\{\theta_\alpha\}_{\alpha=1}^m$ only through their empirical
distribution.
Such observations motivate the mean-field viewpoint, in which the empirical
measure $\mu_m$ is regarded as the fundamental state variable describing the
network \citep{Mei2018meanfieldview,chizat2018globalconvergencegradientdescent}.

\paragraph{Initialization and weak convergence.}
Assume that the parameters $\{\theta_\alpha(0)\}_{\alpha=1}^m$ are initialized
i.i.d.\ according to a probability measure $\mu_0$ on $\Omega$.
Define the empirical measure at initialization
\[
	\mu_{m,0}
	:=
	\frac1m\sum_{\alpha=1}^m \delta_{\theta_\alpha(0)} .
\]
Then $\mu_{m,0}$ converges weakly to $\mu_0$ almost surely as $m\to\infty$.
Indeed, for any bounded continuous test function $\psi:\Omega\to\mathbb R$,
\[
	\int_\Omega \psi\,d\mu_{m,0}
	=
	\frac1m\sum_{\alpha=1}^m \psi(\theta_\alpha(0))
	\;\xrightarrow[m \to \infty]{\text{a.s.}}\;
	\mathbb E_{\theta\sim\mu_0}[\psi(\theta)]
	=
	\int_\Omega \psi\,d\mu_0,
\]
where the convergence follows from the strong law of large numbers applied to
the i.i.d.\ random variables $\psi(\theta_\alpha(0))$.
Since this holds for all bounded continuous $\psi$, we conclude that
$\mu_{m,0} \to \mu_0$ \citep{vadaranjan1958convergenceofprobabilitydistribution}.
As a consequence, the network output at initialization converges pointwise to
the deterministic limit
\[
	f_{\mu_0}(x)
	:=
	\int_\Omega v\,\varphi(w^\top x)\,d\mu_0(v,w).
\]
For instance, under i.i.d.\ Gaussian initialization of the parameters, i.e.
$v_0 \sim \mathcal{N}(0,I_m)\; W_0 \sim \mathcal{N}(0, I_m \otimes I_d)$, the
limiting initial measure $\mu_0 = \mathcal{N}(0, I_{d+1})$ is a Gaussian
measure on a $d+1$-dimensional space.


\subsection{Training dynamics and evolution of the empirical measure}

Let $X=(x_1,\dots,x_n)\in\mathbb R^{d\times n}$ and
$y=(y_1,\ldots,y_n)\in\mathbb R^n$ denote the training set, and consider the
squared loss
\[
	L(v,W)
	=
	\frac12\sum_{i=1}^n\bigl(f_{v,W}(x_i)-y_i\bigr)^2
	=
	\frac12\|f_{v,W}(X)-y\|_2^2,
\]
where $f_{v,W}(X):=(f_{v,W}(x_i))_{i=1}^n$.

We study continuous-time gradient flow for a finite-width network with $m$
neurons and learning rate $\eta>0$,
\[
	\dot v_{t,\alpha}
	=
	-\eta\,\nabla_{v_\alpha}L(v_t,W_t),
	\qquad
	\dot w_{t,\alpha}
	=
	-\eta\,\nabla_{w_\alpha}L(v_t,W_t),
	\qquad
	\alpha\in\{1,\dots,m\}.
\]

A direct computation yields, for each $\alpha$,
\begin{align}
	\dot v_{t, \alpha}
	 & =
	\frac{\eta}{m}\,
	\varphi(X^\top w_{t,\alpha})^\top
	\bigl(y-f_{v_t,W_t}(X)\bigr),
	\label{eq:mf-ode-v} \\
	\dot w_{t,\alpha}
	 & =
	\frac{\eta}{m}\,
	X\Bigl(
	\varphi'(X^\top w_{t,\alpha})
	\odot
	\bigl(y-f_{v_t,W_t}(X)\bigr)
	\Bigr)v_{t,\alpha},
	\label{eq:mf-ode-w}
\end{align}
where $\odot$ denotes elementwise (Hadamard) multiplication.

As seen in \eqref{eq:mf-ode-v}--\eqref{eq:mf-ode-w}, evolution of
each neuron depends on the
parameters only through the current empirical measure $\mu_{m,t}$ via $f_{v_t, W_t}(X)$.
Thus, the particle system \eqref{eq:mf-ode-v}--\eqref{eq:mf-ode-w} induces an
evolution of the empirical measure in parameter space
\citep{Mei2018meanfieldview,Rotskoff_2022}.

Hence, the particle system \eqref{eq:mf-ode-v}--\eqref{eq:mf-ode-w} can be viewed as
an interacting particle system driven by a measure-dependent velocity field.
For any $\theta=(v,w)\in\Omega$ and any $\mu\in\mathcal P(\Omega)$, define
\[
	b((v,w);\mu)
	:=
	\begin{pmatrix}
		\varphi(X^\top w)^\top\bigl(y-f_\mu(X)\bigr) \\[2mm]
		X\Bigl(\varphi'(X^\top w)\odot\bigl(y-f_\mu(X)\bigr)\Bigr)\,v
	\end{pmatrix}
\]
Then the gradient-flow dynamics can be written compactly as
\[
	\dot\theta_{t,\alpha}
	=
	\frac{\eta}{m}\,b(\theta_{t,\alpha};\mu_{m,t}),
	\qquad \alpha=1,\dots,m.
\]
In particular, each particle interacts with the rest of the system only through the
current empirical measure $\mu_{m,t}$ (via $f_{\mu_{m,t}}(X)$).

Equivalently, $\mu_{m,t}$ solves the continuity equation
\citep{Rotskoff_2022,chizat2018globalconvergencegradientdescent}.
\begin{equation}
	\partial_t\mu_{m,t}
	+
	\nabla_\theta\!\cdot\!\Bigl(\mu_{m,t}\,\frac{\eta}{m}\,b(\cdot;\mu_{m,t})\Bigr)
	=
	0,
	\label{eq:mf-continuity-compact}
\end{equation}
in the sense of distributions, i.e.\ for every test function $\psi\in C_c^\infty(\Omega)$,
\[
	\frac{d}{dt}\int_\Omega \psi(\theta)\,d\mu_{m,t}(\theta)
	=
	\frac{\eta}{m}\int_\Omega \nabla_\theta\psi(\theta)\cdot b(\theta;\mu_{m,t})\,d\mu_{m,t}(\theta).
\]
Choosing the mean-field scaling $\eta=m$ yields a nontrivial $\mathcal O(1)$ evolution in time and,
as $m\to\infty$, one expects $\mu_{m,t} \to \mu_t$, where the limit $\mu_t$ satisfies
\citep{Mei2018meanfieldview,Rotskoff_2022,golikov2025deepneuralnetworkslargewidthbehavior,chizat2018globalconvergencegradientdescent}.
\begin{equation}
	\partial_t\mu_t+\nabla_\theta\!\cdot\bigl(\mu_t\,b(\cdot;\mu_t)\bigr)=0,
	\qquad \mu_0=\mathcal N (0,I_{d+1}),
	\label{eq:mf-continuity-with-initial-conditions}
\end{equation}

\paragraph{Push-forward formulation.}
Recall that if $g:\mathcal X\to\mathcal Z$ is measurable and $\mu\in\mathcal P(\mathcal X)$, then the
\emph{push-forward} of $\mu$ by $g$ is the measure $g_*\mu\in\mathcal P(\mathcal Z)$ defined by
\[
	(g_*\mu)(A):=\mu\bigl(g^{-1}(A)\bigr),
	\qquad A\subset\mathcal Z \text{ measurable}.
\]
Let $\Phi_t:\Omega\to\Omega$ denote the flow map associated with the velocity field
$\frac{\eta}{m}\,b(\cdot;\mu_{m,t})$. Then the empirical measure is transported by this flow:
\[
	\mu_{m,t} = (\Phi_t)_*\mu_{m,0}.
\]
In particular, for any measurable observable $h:\Omega\to\mathbb R$,
\[
	\int_\Omega h(\theta)\,d\mu_{m,t}(\theta)
	=
	\int_\Omega h\!\left(\Phi_t(\theta)\right)\,d\mu_{m,0}(\theta),
\]
so evaluating the network at time $t$ is obtained by taking $h_{x}(\theta):=v\,\varphi(w^\top x)$ and writing
\[
	f_{\mu_{m,t}}(x)=\int_\Omega v\,\varphi(w^\top x)\,d\mu_{m,t}(v,w)
	=\int_\Omega v\,\varphi(w^\top x)\,d\bigl((\Phi_t)_*\mu_{m,0}\bigr)(v,w).
\]

\paragraph{Advantages over the NTK limit.}
Recall the finite-width NTK expression \eqref{eq:ntk-1hl-finite}. In the mean-field
parameterization $f(x)=\frac1m\sum_{\alpha=1}^m v_\alpha\varphi(w_\alpha^\top x)$, the empirical tangent
kernel is
\begin{equation}
	\Theta^{(m)}_t(x,x')
	=
	\frac{1}{m^2}\sum_{\alpha=1}^m
	\Bigl(
	\varphi(w_{t,\alpha}^\top x)\,\varphi(w_{t,\alpha}^\top x')
	+
	v_{t,\alpha}^2\,\varphi'(w_{t,\alpha}^\top x)\,\varphi'(w_{t,\alpha}^\top x')\,x^\top x'
	\Bigr).
	\label{eq:mf-ntk-equation}
\end{equation}
Assuming the weights remain $\mathcal O(1)$ as $m\to\infty$, each coordinate of $\nabla_\theta f_t(x)$ is
$\mathcal O(1/m)$, so $\Theta^{(m)}_t(x,x')=\langle\nabla_\theta f_t(x),\nabla_\theta f_t(x')\rangle$ is a sum
of $m$ terms of size $\mathcal O(m^{-2})$, hence $\Theta^{(m)}_t(x,x')=\mathcal O(m^{-1})$.
Moreover, for the squared loss, differentiating the predictions along gradient flow yields
\[
	\partial_t f_t(X)
	=
	-\eta\,\Theta^{(m)}_t(X,X)\bigl(f_t(X)-y\bigr),
\]
so the effective operator driving learning is the \emph{scaled} kernel $\eta\,\Theta^{(m)}_t$. The mean-field
time scaling $\eta=m$ therefore produces an $\mathcal O(1)$ evolution and yields a finite kernel limit:
\[
	\lim_{m\to\infty}\bigl(\eta\,\Theta^{(m)}_t(x,x')\bigr)
	=
	\int_\Omega
	\Bigl(
	\varphi(w^\top x)\,\varphi(w^\top x')
	+
	v^2\,\varphi'(w^\top x)\,\varphi'(w^\top x')\,x^\top x'
	\Bigr)\,d\mu_t(v,w).
\]
As $\mu_t$ evolves according to the mean-field transport equation, this limiting kernel also evolves in time,
capturing feature learning beyond the frozen-kernel NTK regime \citep{golikov2025deepneuralnetworkslargewidthbehavior}.

\paragraph{Disadvantages.}
The mean-field scaling $\eta=m$ yields a deterministic measure-valued evolution
$(\mu_t)_{t\ge 0}$ as $m\to\infty$, governed by the nonlinear continuity
equation \eqref{eq:mf-continuity-with-initial-conditions}. However, its
long-time behavior is generally difficult to characterize: it is not clear in
general whether $\mu_t$ converges as $t\to\infty$, nor which minimizer (if any)
is selected at the level of measures. Nevertheless, when the dynamics is
interpreted as a Wasserstein gradient flow, there exist results proving global
convergence of the loss $L$ under suitable assumptions
\citep{Mei2018meanfieldview,chizat2018globalconvergencegradientdescent}.
Moreover, unlike the NTK limit, extending the above measure-evolution
construction beyond two-layer networks is not straightforward. The main
difficulty is that for deep networks the hidden units are no longer
permutation-invariant in a way that yields a simple empirical measure on a
fixed parameter space. Several multilayer mean-field-type constructions have
been proposed to address this, but they are typically more involved than the
two-layer case \citep{nguyen2019meanfieldlimitlearning,araujo2019meanfieldlimitcertaindeep,
	sirignano2019meanfieldanalysisneural,nguyen2023rigorousframeworkmeanfield}.


\section{Spectral bias}\label{sec:spectral-bias}

A striking and widely reported phenomenon in neural network training is that
gradient-based optimization does not fit all components of a target function at
the same rate. Instead, networks tend to learn ``simple'' structure first and
refine fine-scale or highly oscillatory structure later. This preference is
commonly referred to as \emph{spectral bias} (or the \emph{frequency principle}).
In its classical formulation, spectral bias asserts that, when a target function
is decomposed into Fourier modes, lower-frequency components are learned
earlier and at a faster rate than higher-frequency components
\citep{rahaman2019spectralbiasneuralnetworks}.

In this section we briefly review empirical evidence for spectral bias and
summarize theoretical viewpoints that relate it to the spectrum of the operator
governing training dynamics in the kernel (NTK) regime. We also highlight
extensions that study how the phenomenon depends on the input distribution and
how it manifests outside the training set.

\subsection{Empirical evidence and experimental protocols}

A standard way to empirically probe spectral bias is to choose a target with a
controlled frequency decomposition and to track the frequency content of the
network's prediction $f_t$ during training.

\paragraph{Synthetic Fourier regression.}
A canonical experiment is 1D regression on a target constructed as a sum of
sinusoids,
\[
	y(z)=\sum_{i=1}^r A_i \sin(2\pi k_i z + \phi_i),
	\qquad z\in[0,1],
\]
and monitoring the discrete Fourier spectrum of the learned predictor as a
function of training time. \citet{rahaman2019spectralbiasneuralnetworks} report
that, for deep ReLU networks trained by (full-batch) gradient descent, Fourier
coefficients at smaller frequencies $k_i$ grow substantially earlier than those
at larger $k_i$, even when the amplitudes are matched or when high-frequency
components have larger amplitude \citep{rahaman2019spectralbiasneuralnetworks}.
This ``frequency-dependent learning speed'' is one of the most direct empirical
signatures of spectral bias.

\paragraph{Robustness and perturbation tests.}
A complementary protocol is to train to near-zero training error and then apply
random perturbations in parameter space, comparing how the Fourier spectrum of
the realized function changes. \citet{rahaman2019spectralbiasneuralnetworks}
observe that lower-frequency components of the learned function are
substantially more robust to such perturbations than higher-frequency
components, suggesting that the network parameterization itself represents low
frequencies in a more stable manner \citep{rahaman2019spectralbiasneuralnetworks}.

\paragraph{Distributional effects and ``local'' frequency learning.}
Spectral bias is often presented under uniformly sampled inputs, but empirical
studies also examine how it interacts with the input distribution. For example,
\citet{basri2020frequencybiasneuralnetworks} compare learning under uniform and
non-uniform sampling densities and show that the ordering ``low frequencies
before high frequencies'' can be modulated by where samples concentrate: dense
regions of the input space may exhibit faster learning of higher-frequency
structure than sparse regions \citep{basri2020frequencybiasneuralnetworks}.
This line of experimentation motivates viewing spectral bias as a property of an
operator defined jointly by the model and the data distribution, rather than as
a purely architectural effect.

\subsection{Operator spectrum viewpoint and the NTK regime}

A common theoretical explanation of spectral bias proceeds by identifying an
operator whose eigendecomposition controls learning rates. In the kernel/NTK
regime, training dynamics are approximately linear in function space and
decompose along eigendirections of the corresponding kernel operator. In this
setting, spectral bias can be interpreted more generally as a bias toward
learning the leading eigenfunctions of the kernel (rather than specifically
Fourier modes) \citep{bowman2022spectralbiasoutsidetraining}.

\paragraph{Decomposition along NTK eigenfunctions.}
\citet{cao2020understandingspectralbiasdeep} give a rigorous account of this
mechanism in the NTK regime: the training process can be decomposed along
eigenfunctions of the NTK operator, with each component converging at a rate
determined by the associated eigenvalue. As a consequence, components aligned
with larger eigenvalues are learned faster, yielding a principled notion of
``spectral preference'' \citep{cao2020understandingspectralbiasdeep}.

In general, the eigenfunctions of the NTK operator depend jointly on
the input distribution and the kernel, so they need not coincide with the
standard Fourier basis. A simplification occurs in the \emph{infinite-width}
(kernel) limit, where the empirical NTK concentrates around a deterministic
limit $\bar\Theta$ given by an expectation over the random initialization.

To illustrate how Fourier modes arise, consider inputs on the circle $S^1$,
parameterized by $\theta\in[0,2\pi)$ and embedded as
$x(\theta)=(\cos\theta,\sin\theta)\in\mathbb R^2$. Assume isotropic
initialization (e.g.\ Gaussian), so that $R^\top w$ has the same distribution as $w$ for every
orthogonal matrix $R$. Writing $\bar\Theta$ as an expectation (cf.\
Eq.~\eqref{eq:ntk-1hl-infinite}) and using this rotational invariance in
distribution implies
\[
	\bar\Theta(Rx,Rx')=\bar\Theta(x,x') \qquad \text{for all orthogonal }R.
\]
Since $x(\theta+\alpha)=R_\alpha x(\theta)$ and
$x(\theta)^\top x(\theta')=\cos(\theta-\theta')$, it follows that
$\bar\Theta(\theta,\theta')$ depends only on $\theta-\theta'$; equivalently,
there exists $\kappa$ such that
\[
	\bar\Theta(\theta,\theta')=\kappa(\theta-\theta').
\]

\paragraph{Convolution form on $S^1$.}
Let $\rho$ be the uniform probability measure on $S^1$,
$d\rho(\theta')=d\theta'/(2\pi)$, and define the integral operator
\[
	(T_{\bar\Theta} g)(\theta)
	:=
	\int_0^{2\pi} \bar\Theta(\theta,\theta')\,g(\theta')\,\frac{d\theta'}{2\pi}.
\]
Using $\bar\Theta(\theta,\theta')=\kappa(\theta-\theta')$ and the change of
variables $u=\theta-\theta'$, we obtain
\[
	(T_{\bar\Theta} g)(\theta)
	=
	\int_0^{2\pi} \kappa(u)\,g(\theta-u)\,\frac{du}{2\pi}
	=: (\kappa * g)(\theta),
\]
which is the (circular) convolution operator on $S^1$.

\paragraph{Fourier eigenfunctions and frequency ordering.}
Because $T_{\bar\Theta}$ is a convolution, the Fourier modes
$\phi_q(\theta)=e^{iq\theta}$, $q\in\mathbb Z$, are the eigenfunctions:
\[
	(T_{\bar\Theta}\phi_q)(\theta)
	=
	e^{iq\theta}\int_0^{2\pi}\kappa(u)e^{-iqu}\,\frac{du}{2\pi}
	=
	\lambda_q\,\phi_q(\theta),
	\qquad
	\lambda_q
	=
	\int_0^{2\pi}\kappa(u)e^{-iqu}\,\frac{du}{2\pi}.
\]
If $\kappa$ is real and even (as in $\kappa(u)=\psi(\cos u)$), then
$\lambda_q\in\mathbb R$ and $\lambda_q=\lambda_{-q}$, and one may equivalently
use the real basis $\{\cos(q\theta),\sin(q\theta)\}$.
In kernel gradient dynamics, each mode contracts independently at a rate set by
$\lambda_q$ (e.g.\ $(1-\eta\lambda_q)^t$ in discrete time), so modes with larger
eigenvalues are learned faster. In symmetric NTK settings on the
circle/sphere, the eigenvalues decrease with frequency, leading to the
characteristic ``low frequencies first'' behavior
\citep{basri2020frequencybiasneuralnetworks,rahaman2019spectralbiasneuralnetworks}.

\subsection{Uniform vs.\ non-uniform sampling}\label{subsec:uniform-nonuniform}

The Fourier-mode picture above relies not only on a shift-invariant kernel but
also on the \emph{uniform} input distribution on $S^1$. Under uniform sampling,
$d\rho(\theta')=d\theta'/(2\pi)$, shift invariance $\bar\Theta(\theta,\theta')
	=\kappa(\theta-\theta')$ implies that the integral operator is a circular
convolution and therefore diagonalizes in the Fourier basis. In this setting,
spectral bias can be stated directly in terms of frequency: learning rates are
controlled by the eigenvalues $\lambda_q$ associated with Fourier modes.

Under \emph{non-uniform} sampling, this simplification breaks. If inputs are
distributed according to a density $p(\theta)$ on $S^1$, then the relevant
population operator becomes
\[
	(T_{p}g)(\theta)
	:=
	\int_0^{2\pi} \kappa(\theta-\theta')\,g(\theta')\,p(\theta')\,\frac{d\theta'}{2\pi}.
\]
Even when $\kappa$ depends only on $\theta-\theta'$, the extra factor
$p(\theta')$ destroys shift invariance, so $T_p$ is no longer a pure convolution
operator and its eigenfunctions need not be the global Fourier modes. As a
result, a Fourier harmonic target can project onto multiple eigendirections of
$T_p$, and ``frequency'' becomes distribution-dependent: the modes learned
early are those aligned with the \emph{top eigenfunctions of the density-weighted
	operator} rather than the lowest Fourier frequencies.

This dependence on $p$ is analyzed explicitly by \citet{basri2020frequencybiasneuralnetworks},
who study the kernel regime on the circle/sphere and show that non-uniform
densities can produce eigenfunctions with localized oscillatory structure and
learning behavior that varies across the input space, with denser regions
exhibiting effectively faster learning of finer-scale structure
\citep{basri2020frequencybiasneuralnetworks}.

\subsection{Spectral bias beyond the training set}

Most early demonstrations of spectral bias are reported on the training set via
Fourier spectra or projections onto empirical kernel eigenvectors. A natural
question is whether a comparable phenomenon holds in function space more
globally. \citet{bowman2022spectralbiasoutsidetraining} address this in the
kernel regime by proving bounds that compare finite-width training trajectories
to idealized kernel dynamics and conclude that networks inherit the bias of the
NTK integral operator over the input space, not merely at the training samples.
In particular, they argue that eigenfunctions of the NTK integral operator are
learned at rates corresponding to their eigenvalues, providing a mechanism for
spectral bias that persists outside the training set \citep{bowman2022spectralbiasoutsidetraining}.

\subsection{Discussion and connection to feature learning}

The operator-spectrum viewpoint provides a clean account of spectral bias in
kernelized regimes, where learning rates are dictated by the spectrum of a fixed
(or nearly fixed) operator. Outside the NTK regime, however, features can evolve
substantially during optimization, and the relevant operator (e.g.\ the tangent
kernel) may drift. From this perspective, spectral bias can be studied both as a
\emph{property of the initial linearized dynamics} and as a phenomenon that may
interact with feature learning through time-dependent spectral structure. This
interaction motivates empirical analyses that track mode-wise error decay
alongside kernel evolution and provides a natural link between kernel-based and
mean-field viewpoints.

\section{Research questions}\label{sec:rq}

On symmetric domains such as $S^1$ under uniform sampling, the infinite-width
neural tangent kernel (NTK) induces a convolution operator whose eigenfunctions
are the Fourier modes. In this idealized setting, gradient-based training
dynamics admit a closed-form spectral description: residual components along
distinct Fourier modes evolve independently at rates determined by the kernel
spectrum. The preliminary experiments in Chapter~\ref{cha:prelim} reproduce this
frequency ordering, but also exhibit systematic deviations at finite width and
finite sample size, including non-monotone mode amplitudes and slow transients
that persist even when global kernel drift appears small. These observations
suggest that spectral bias in practice cannot be fully explained by the ideal
convolutional NTK model and motivate a perturbative analysis beyond this limit.

\paragraph{Main research question.}
How does spectral bias in finite-width ReLU networks trained on uniformly sampled
inputs on $S^1$ deviate from the convolutional NTK prediction, and how can these
deviations be characterized through finite-sample and finite-width perturbations
of the kernel operator governing training dynamics?

\paragraph{Subquestions.}
\begin{enumerate}
	\item[(Q1)] \textbf{Empirical separation of finite-width and finite-sample effects.}
	      How do mode-wise residual dynamics change when network width and sample size are
	      varied independently in Fourier-mixture regression on $S^1$?

	      This question aims to identify which deviations from ideal Fourier-mode decay
	      persist as width increases at fixed sample size and which vanish as the number
	      of uniformly sampled training points increases at fixed large width.

	\item[(Q2)] \textbf{Finite-sample perturbation of the convolutional NTK.}
	      At infinite width, how does replacing the population convolution operator on
	      $S^1$ by its empirical discretization based on $n$ uniformly sampled points
	      perturb its Fourier eigenvalues and eigenfunctions?

	      More precisely, this question targets bounds or asymptotic estimates for the
	      deviation between the population NTK integral operator $T$ and its empirical
	      counterpart $T_n$, and an analysis of how these perturbations induce coupling
	      between Fourier modes as a function of sample size and frequency.

	\item[(Q3)] \textbf{Finite-width perturbation of the NTK operator.}
	      In the infinite-population setting (uniform measure on $S^1$), how does finite
	      width perturb the spectrum and eigenspaces of the NTK operator at initialization
	      and during early training?

	      This question focuses on treating the finite-width NTK as a random perturbation
	      of the deterministic convolutional operator and on characterizing eigenspace
	      rotation or instability as potential mechanisms behind non-monotone residual
	      dynamics and slow transients observed in practice.
\end{enumerate}

\paragraph{Planned schedule (12--15 weeks).}
The project is structured to ensure that empirical results are obtained early
and used to guide the theoretical analysis.

\begin{itemize}
	\item \textbf{Weeks 1--3: Empirical analysis (Q1).}
	      Extend the existing experimental setup to independently vary network width and
	      sample size (e.g. sample size $n \in \{250, 500, 1000, 5000,\ldots\}$). Produce mode-wise residual decay
	      plots and kernel eigenspace diagnostics to isolate finite-width and finite-sample
	      effects.

	\item \textbf{Weeks 4--8: Finite-sample theory (Q2).}
	      Formalize the convolutional NTK integral operator on $S^1$ and its empirical
	      discretization. Derive bounds or asymptotic estimates for eigenvalue and
	      eigenfunction perturbations and relate them to observed Fourier-mode mixing.

	\item \textbf{Weeks 9--12: Finite-width analysis (Q3).}
	      Study finite-width fluctuations of the NTK under uniform population measure.
	      Analyze random perturbations of the convolutional operator and develop a
	      perturbative explanation for eigenspace rotation and non-monotone residual
	      dynamics.
\end{itemize}
