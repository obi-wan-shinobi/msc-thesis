\section{Formalization of the Neural Tangent Kernel (NTK)}\label{sec:ntk-formalization}

\subsection{Setup and Assumptions}

We consider a supervised learning setting with data $\{(x_i,y_i)\}_{i=1}^n$,
where $x_i \in \mathbb{R}^d$ and $y_i \in \mathbb{R}$. A neural network
$f:\mathbb{R}^d \times \Theta \to \mathbb{R}$ is parameterized by $\theta \in
	\Theta \subseteq \mathbb{R}^p$, where $p$ is the number of parameters.

\begin{assumption}[Model and training]\label{ass:model}
	\leavevmode
	\begin{enumerate}
		\item (\emph{Differentiability}) The network $f(x ;\theta)$ is differentiable in $\theta$.
		\item (\emph{Random initialization}) We initialize parameters
		      $\theta_0$ i.i.d.\ with zero mean and variance scaled according to
		      layer input dimensions (e.g.\ NTK schemes), so that activations and
		      gradients remain well-behaved as depth/width grow
		      \citep{neal1996priors, jacot2018ntk, lee2020wide}.
		\item (\emph{Training})
		      We train $f$ by gradient descent on the squared loss:
		      \[
			      L(\theta) = \frac{1}{2}\sum_{i=1}^n \big(f(x_i;\theta) - y_i\big)^2.
		      \]
	\end{enumerate}
\end{assumption}


\subsection{Linearization around Initialization}

A first-order Taylor expansion of $f(x;\theta)$ around $\theta_0$ gives
\[
	f(x;\theta) \;\approx\; f(x;\theta_0) + \nabla_\theta f(x;\theta_0)^\top (\theta - \theta_0).
\]
\begin{itemize}
	\item $f(x;\theta_0)$ is the network output at initialization (a bias term).
	\item $\phi(x) := \nabla_\theta f(x;\theta_0)$ is the feature vector induced at initialization.
\end{itemize}
Thus, locally, the network behaves as a linear model in $\theta$:
\begin{equation}
	f(x;\theta) \approx f(x;\theta_0) + \phi(x)^\top (\theta - \theta_0).
	\label{eq:ntk}
\end{equation}

\subsection{Neural Tangent Kernel}

\begin{definition}[Neural Tangent Kernel {\citep{jacot2018ntk}}]
	Given initialization $\theta_0$, the Neural Tangent Kernel (NTK) is
	\[
		K(x,x') \;=\; \nabla_\theta f(x;\theta_0)^\top \nabla_\theta f(x';\theta_0).
	\]
\end{definition}

\noindent
The NTK captures how parameter updates couple the outputs of $x$ and $x'$.

\begin{remark}\label{rem:ntk}
	\leavevmode
	\begin{itemize}
		\item $K(x,x')$ is positive semidefinite \citep{jacot2018ntk}.
		\item In the infinite-width limit, under common initializations, $K(x,x')$ converges almost surely to a deterministic kernel depending only on architecture and activation \citep{jacot2018ntk, lee2020wide}.
		\item For finite but large width, $K$ is still a random kernel due to random initialization,
		      but it concentrates around its infinite-width expectation.
		      Fluctuations vanish at rate $O(1/\sqrt{m})$ as width $m \to \infty$
		      \citep{jacot2018ntk, lee2020wide}.
	\end{itemize}
\end{remark}

\subsection{NTK Characterization (one hidden layer, no biases)}\label{subsec:ntk-onelayer}

Consider a width-$m$ one-hidden-layer network
\[
	f(x)\;=\;\frac{1}{\sqrt{m}}\sum_{r=1}^m a_r\,\sigma(w_r^\top x),
\]
with parameters $\theta=\{(a_r,w_r)\}_{r=1}^m$, activation $\sigma$, and random initialization
\[
	a_r \sim \mathcal N(0,\sigma_a^2),\qquad
	w_r \sim \mathcal N\!\Big(0,\frac{\sigma_w^2}{d}I_d\Big),
\]
independently across $r$.

\subsubsection*{Finite-width NTK at initialization}
By definition,
\[
	K_m(x,x') \;=\; \nabla_\theta f(x)^\top \nabla_\theta f(x')
	\;=\;\sum_{r=1}^m\!\left[
		\underbrace{\nabla_{a_r} f(x)\,\nabla_{a_r} f(x')}_{\text{output-weight part}}
		\;+\;
		\underbrace{\nabla_{w_r} f(x)^\top \nabla_{w_r} f(x')}_{\text{hidden-weight part}}
		\right].
\]
Compute the gradients:
\[
	\nabla_{a_r} f(x)=\frac{1}{\sqrt{m}}\sigma(w_r^\top x),
	\qquad
	\nabla_{w_r} f(x)=\frac{1}{\sqrt{m}}\,a_r\,\sigma'(w_r^\top x)\,x.
\]
Hence,
\[
	\boxed{\;
		K_m(x,x')
		=\frac{1}{m}\sum_{r=1}^m \sigma(w_r^\top x)\,\sigma(w_r^\top x')
		+\frac{1}{m}\sum_{r=1}^m a_r^2\,\sigma'(w_r^\top x)\,\sigma'(w_r^\top x')\,x^\top x'\; }.
\]

\subsubsection*{Infinite-width limit}
The two sums are empirical averages of i.i.d.\ terms. Since $a_r$ and $w_r$ are independent with finite moments,
the (strong) law of large numbers gives, almost surely,
\[
	\frac{1}{m}\sum_{r=1}^m \sigma(w_r^\top x)\,\sigma(w_r^\top x')
	\;\longrightarrow\;
	\mathbb E_{w}\big[\sigma(w^\top x)\,\sigma(w^\top x')\big],
\]
\[
	\frac{1}{m}\sum_{r=1}^m a_r^2\,\sigma'(w_r^\top x)\,\sigma'(w_r^\top x')
	\;\longrightarrow\;
	\sigma_a^2\,\mathbb E_{w}\big[\sigma'(w^\top x)\,\sigma'(w^\top x')\big].
\]
Thus, in the infinite-width limit, the empirical NTK converges almost surely to a deterministic kernel
\citep{jacot2018ntk, lee2020wide}.
\[
	\boxed{\;
		K_\infty(x,x')
		=\mathbb E_{w}\!\big[\sigma(w^\top x)\,\sigma(w^\top x')\big]
		\;+\;
		\sigma_a^2\,x^\top x'\,\mathbb E_{w}\!\big[\sigma'(w^\top x)\,\sigma'(w^\top x')\big]}
	\;
\]
with $w\sim \mathcal N(0,\frac{\sigma_w^2}{d}I_d)$.

% \begin{remark}
% 	Without the $1/d$ factor in $w$'s covariance, the variance of $w^\top x$ would scale like $\|x\|^2 d$, making the expectations above diverge or degenerate. The $1/d$ scaling keeps $\Var(w^\top x)=\frac{\sigma_w^2}{d}\|x\|^2=O(1)$, so the expectations are well-defined and the LLN applies cleanly.
% \end{remark}

\subsection{One Hidden Layer: Adding Biases}\label{subsec:ntk-onelayer-bias}

If we allow per-neuron biases, we get minimal changes from \autoref{subsec:ntk-onelayer}.
Consider
\[
	f(x)\;=\;\frac{1}{\sqrt{m}}\sum_{r=1}^m a_r\,\sigma\!\big(w_r^\top x + b_r\big),
	\quad
	a_r \sim \mathcal N(0,\sigma_a^2),\;
	w_r \sim \mathcal N\!\Big(0,\tfrac{\sigma_w^2}{d}I_d\Big),\;
	b_r \sim \mathcal N(0,\sigma_b^2),
\]
independently across $r$. Let $u_r(x):=w_r^\top x + b_r$.

\paragraph{Finite width}

Gradients are
\[
	\nabla_{a_r} f(x)=\tfrac{1}{\sqrt{m}}\sigma(u_r(x)),\quad
	\nabla_{w_r} f(x)=\tfrac{1}{\sqrt{m}}\,a_r\,\sigma'(u_r(x))\,x,\quad
	\nabla_{b_r} f(x)=\tfrac{1}{\sqrt{m}}\,a_r\,\sigma'(u_r(x)).
\]
Thus
\[
	\boxed{\;
		K_m(x,x')
		=\frac{1}{m}\sum_{r=1}^m \sigma(u_r(x))\,\sigma(u_r(x'))
		+\frac{1}{m}\sum_{r=1}^m a_r^2\,\sigma'(u_r(x))\,\sigma'(u_r(x'))\,\big(x^\top x' + 1\big)\; }.
\]

\paragraph{Infinite width (preactivation covariance picks up $\sigma_b^2$).}

With $(U,V)$ jointly Gaussian:
\[
	\Var(U)=\tfrac{\sigma_w^2}{d}\|x\|^2+\sigma_b^2,\quad
	\Var(V)=\tfrac{\sigma_w^2}{d}\|x'\|^2+\sigma_b^2,\quad
	\mathrm{Cov}(U,V)=\tfrac{\sigma_w^2}{d}x^\top x'+\sigma_b^2,
\]
we have
\[
	\boxed{\;
		K_\infty(x,x')
		=\mathbb E\!\big[\sigma(U)\sigma(V)\big]
		+\sigma_a^2\,\big(x^\top x' + 1\big)\,\mathbb E\!\big[\sigma'(U)\sigma'(V)\big]\; }.
\]

\subsection{Extension to Deep Fully--Connected Networks}\label{subsec:deep-ntk}

\paragraph{No biases. }
Let $n_0=d,\ n_1,\ldots,n_{L-1}$ be layer widths and define
\[
	\alpha^{(0)}(x)=x,\qquad
	\tilde\alpha^{(\ell+1)}(x)=W^{(\ell)}\alpha^{(\ell)}(x),\qquad
	\alpha^{(\ell+1)}(x)=\sigma\!\big(\tilde\alpha^{(\ell+1)}(x)\big),
\]
for $\ell=0,\dots,L-2$, with entries $W^{(\ell)}\stackrel{\text{i.i.d.}}{\sim}\mathcal N\!\Big(0,\frac{\sigma_w^2}{n_\ell}\Big)$.
The scalar output is
\[
	f_\theta(x)=\frac{1}{\sqrt{n_{L-1}}}\sum_{r=1}^{n_{L-1}} a_r\,\alpha^{(L-1)}_r(x),\qquad a_r\sim\mathcal N(0,\sigma_a^2).
\]

\medskip\noindent
Define the activation covariance
\[
	\Sigma^{(\ell)}(x,x'):=\mathbb E\!\big[\alpha^{(\ell)}_r(x)\,\alpha^{(\ell)}_r(x')\big],\qquad
	q^{(\ell)}(x):=\Sigma^{(\ell)}(x,x),
\]
with base $\Sigma^{(0)}(x,x')=\tfrac{1}{d}\,x^\top x'$.
Then the forward (NNGP) recursion is
\[
	\boxed{\;
		\Sigma^{(\ell+1)}(x,x')
		=\mathbb E_{f\,\sim\,\mathcal{GP}\!\big(0,\ \sigma_w^2\,\Sigma^{(\ell)}\big)}\!
		\big[\sigma\!\big(f(x)\big)\,\sigma\!\big(f(x')\big)\big]\;}.
\]
Define the derivative–correlation kernel
\[
	\boxed{\;
		\dot\Sigma^{(\ell+1)}(x,x')
		:=\mathbb E_{f\,\sim\,\mathcal{GP}\!\big(0,\ \sigma_w^2\,\Sigma^{(\ell)}\big)}\!
		\big[\sigma'\!\big(f(x)\big)\,\sigma'\!\big(f(x')\big)\big]\; }.
\]
The limiting NTK satisfies
\[
	\boxed{\;
		\Theta_\infty^{(1)}(x,x')=\Sigma^{(1)}(x,x'),\qquad
		\Theta_\infty^{(\ell+1)}(x,x')
		=\Theta_\infty^{(\ell)}(x,x')\,\dot\Sigma^{(\ell+1)}(x,x')+\Sigma^{(\ell+1)}(x,x')\; }.
\]
For a dataset $\{x_i\}$:
\[
	\Theta^{(\ell+1)}=\Theta^{(\ell)}\odot \dot\Sigma^{(\ell+1)}+\Sigma^{(\ell+1)},\qquad
	\Theta^{(1)}=\Sigma^{(1)}.
\]
Setting $L=2$ recovers the two-term one-layer kernel.

% \paragraph{Including biases}
%
% Introduce the \emph{preactivation} covariance
% $Q^{(\ell)}(x,x'):=\mathbb E[\tilde\alpha^{(\ell)}_r(x)\,\tilde\alpha^{(\ell)}_r(x')]$.
% Initialize and recurse
% \[
% 	Q^{(1)}(x,x')=\frac{\sigma_w^2}{d}\,x^\top x' + \sigma_b^2,\qquad
% 	\Sigma^{(\ell)}(x,x')=\mathbb E_{(U,V)\sim\mathcal N(0,Q^{(\ell)}(x,x'))}\![\sigma(U)\sigma(V)],
% \]
% \[
% 	\boxed{\;Q^{(\ell+1)}(x,x')=\sigma_w^2\,\Sigma^{(\ell)}(x,x')+\sigma_b^2,\qquad
% 	\dot\Sigma^{(\ell)}(x,x')=\mathbb E_{(U,V)\sim\mathcal N(0,Q^{(\ell)}(x,x'))}\![\sigma'(U)\sigma'(V)]\; }.
% \]
% The NTK recursion \emph{itself} stays the same:
% $\Theta_\infty^{(\ell+1)}=\Theta_\infty^{(\ell)}\,\dot\Sigma^{(\ell+1)}+\Sigma^{(\ell+1)}$.
% At $L=2$, this reproduces the one-layer bias effects in \autoref{subsec:ntk-onelayer-bias}
% (added constant direction via biases and the $x^\top x' + 1$ factor in the propagated term).

\subsection{Training Dynamics under NTK}

We train with squared loss
\[
	L(\theta)\;=\;\frac{1}{2}\sum_{i=1}^n \big(f(x_i;\theta)-y_i\big)^2,
\]
and consider \emph{gradient flow} in parameter space, i.e.\ the continuous-time limit
of gradient descent as the step size $\eta \to 0$:
\[
	\frac{d\theta_t}{dt}\;=\;-\nabla_\theta L(\theta_t).
\]
Let $f_t(x_i):=f(x_i;\theta_t)$. By the chain rule,
\[
	\frac{d}{dt} f_t(x_i)
	\;=\;\nabla_\theta f(x_i;\theta_t)^\top \frac{d\theta_t}{dt}
	\;=\;-\nabla_\theta f(x_i;\theta_t)^\top \nabla_\theta L(\theta_t).
\]
Compute the parameter gradient of the loss:
\[
	\nabla_\theta L(\theta_t)\;=\;\sum_{j=1}^n \big(f_t(x_j)-y_j\big)\,\nabla_\theta f(x_j;\theta_t).
\]
Substituting gives
\[
	\frac{d}{dt} f_t(x_i)
	\;=\;-\sum_{j=1}^n \underbrace{\nabla_\theta f(x_i;\theta_t)^\top \nabla_\theta f(x_j;\theta_t)}_{=:~K_t(x_i,x_j)}\,
	\big(f_t(x_j)-y_j\big).
\]
Stacking $f_t=(f_t(x_1),\dots,f_t(x_n))$ yields the vector ODE
\[
	\frac{d}{dt} f_t \;=\; -\,K_t\,(f_t - y),
\]
where $[K_t]_{ij}=K_t(x_i,x_j)$ is the (time–dependent) NTK matrix.

\paragraph{Constant–kernel (NTK) regime.}
In the infinite–width limit (or under a lazy–training approximation), the
kernel remains essentially constant during training, $K_t \approx K_0=:K$ \citep{jacot2018ntk}. The
ODE reduces to
\[
	\frac{d}{dt} f_t \;=\; -\,K\,(f_t - y).
\]
Let $r_t:=f_t-y$. Then $\dot r_t=-Kr_t$ with solution $r_t=e^{-Kt}r_0$, i.e.
\[
	f_t \;=\; y + e^{-Kt}\,(f_0-y).
\]

The convergence rate along eigenvector $v_j$ of $K$ is exponential with
rate $\lambda_j$, the corresponding eigenvalue.

%
% Consider the \emph{gradient flow} dynamics, i.e.\ the continuous-time limit
% of gradient descent as the step size $\eta \to 0$. In this regime, the prediction
% vector $f_t = (f(x_1;\theta_t), \dots, f(x_n;\theta_t))$ evolves according to the ODE
% \[
% 	\frac{d}{dt} f_t = -K (f_t - y),
% \]
% where $K \in \mathbb{R}^{n \times n}$ is the deterministic and constant NTK matrix
% with entries $K_{ij} = K(x_i,x_j)$.
%
% \paragraph{Solution.}
% Let $r_t := f_t - y$ denote the residual. Then
% \[
% 	\frac{d}{dt} r_t = -K r_t,\qquad r_0 = f_0 - y,
% \]
% whose solution is
% \[
% 	r_t = e^{-Kt}\, r_0.
% \]
% Hence,
% \[
% 	f_t = y + e^{-Kt}(f_0 - y).
% \]

% \paragraph{Remark.}
% This analysis corresponds to the \emph{continuous-time} limit. For discrete gradient
% descent with step size $\eta > 0$, the dynamics become
% $r_{t+1} = (I - \eta K) r_t$, with solution
% $f_t = y + (I - \eta K)^t (f_0 - y)$, converging under the stability condition
% $\eta < 2/\lambda_{\max}(K)$.

% \paragraph{Alternative derivation (eigendecomposition).}
% Since $K$ is symmetric PSD, write $K = V\Lambda V^\top$. Expanding $r_t=\sum_j c_j(t)v_j$ gives
% $\dot c_j(t) = -\lambda_j c_j(t)$, so $c_j(t)=e^{-\lambda_j t}c_j(0)$ and
% \[
% 	f_t = y + V e^{-\Lambda t} V^\top (f_0 - y) = y + e^{-Kt}(f_0-y).
% \]

\subsection{Lazy Training Regime}

Training is in the \emph{lazy regime} if parameter updates stay small relative to initialization:
\[
	\|\theta_t - \theta_0\| \ll \|\theta_0\|.
\]
Then $\phi(x)$ and the NTK remain essentially constant and training is equivalent to kernel regression with fixed kernel $K$. When $\|\theta_t-\theta_0\|$ is not negligible, $\phi(x)$ evolves, yielding adaptive feature learning beyond NTK.


