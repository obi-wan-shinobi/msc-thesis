\documentclass[a4paper,11pt]{article}
\usepackage[utf8]{inputenc}
\usepackage[T1]{fontenc}
\usepackage{lmodern}
\usepackage{amsmath,amssymb}
\usepackage{graphicx}
\usepackage{caption}
\usepackage{geometry}
\geometry{margin=1in}
\usepackage{hyperref}
\usepackage[round,sort&compress]{natbib}
\hypersetup{colorlinks=true,linkcolor=black,citecolor=black,urlcolor=blue}

% Figures stay in their sections
\usepackage{float}
\usepackage[section]{placeins}

\title{\textbf{EXP002: Empirical NTK Drift and the Onset of the Lazy Regime}\\[2mm]
\large Experiment Summary with Figures}
\author{Shreyas Kalvankar}
\date{November 2025}

\begin{document}
\maketitle

\section*{Goal and Context}

The goal of these experiments is to determine when a finite-width neural network
enters the \emph{lazy regime}, a regime in which the empirical Neural Tangent Kernel (NTK)
remains effectively constant throughout training, and the network dynamics match
those of kernel regression with a fixed NTK.

We investigate whether:
\begin{enumerate}
	\item the empirical NTK becomes approximately constant after some training time;
	\item network predictions match kernel regression when using the frozen NTK;
	\item the NTK freeze time corresponds to a plateau or slowdown in the training loss.
\end{enumerate}

Our regression target is a Fourier mixture on the unit circle:
\[
	f^*(\gamma) = \sum_k a_k \cos(k\gamma + \phi_k),
\]
evaluated on a dense uniform grid, with random subsampled training points.

We sweep fully-connected ReLU MLPs of widths
\[
	W \in \{100,512,1024,2048,10000\}
\]
across multiple seeds.

Snapshots at regular intervals include:
\begin{itemize}
	\item the training--training NTK, $K_{\text{train,train}}(t)$;
	\item the eval--train NTK, $K_{\text{eval,train}}(t)$;
	\item the evaluation-grid network predictions $f_{\text{net}}(\gamma,t)$;
	\item the snapshot training loss $L(t)$;
	\item metadata for seeds and training steps.
\end{itemize}

These allow reconstruction of drift curves, kernel regression predictions, loss evolution,
and freeze-time detection.

% =======================================================
\section{Kernel Drift, Eigenvalue Drift, and Loss Plateau}

\subsection{Kernel Drift Definition}

At each snapshot step we compute the empirical NTK on the training set:
\[
	K(t) := K_{\text{train,train}}(t) \in \mathbb{R}^{M\times M}.
\]

To measure how the kernel evolves during training, we track the
\emph{normalized NTK drift}:
\[
	\Delta_K(t) =
	\frac{\|K(t) - K(0)\|_F}{\|K(0)\|_F}.
\]

This quantity measures the deviation of the kernel from its initialization.
If the network enters the lazy regime, $\Delta_K(t)$ becomes (almost) constant over time.

\begin{figure}[H]
	\centering
	\includegraphics[width=.8\textwidth]{plots/kernel_drift/kernel_drift_low_widths.png}
	\caption{Kernel drift curves for low widths ($W = 100, 512$).}
\end{figure}

\begin{figure}[H]
	\centering
	\includegraphics[width=.8\textwidth]{plots/kernel_drift/kernel_drift_mid_widths.png}
	\caption{Kernel drift for mid-range widths ($W = 1024, 2048$).}
\end{figure}

\begin{figure}[H]
	\centering
	\includegraphics[width=.8\textwidth]{plots/kernel_drift/kernel_drift_high_widths.png}
	\caption{Kernel drift for very wide networks ($W = 10000$).}
\end{figure}


% -------------------------------------------------------
\subsection{Slope-Based Freeze-Time Criterion}

Even when the drift $\Delta_K(t)$ grows slowly, the kernel may have effectively
stopped evolving.
To detect this, we track its \emph{slope}:
\[
	s(t) =
	\frac{\Delta_K(t) - \Delta_K(t - \Delta)}
	{\text{step}(t) - \text{step}(t - \Delta)},
\]
with typical snapshot spacing $\Delta = 2$.

We declare the kernel \emph{frozen} at the earliest time $t_{\text{freeze}}$
for which the drift slope satisfies
\[
	|s(t)| < \varepsilon\,|s(0)|,
\]
for $k$ consecutive snapshot intervals (typically $k=3$ and $\varepsilon=0.01$).

This identifies the point at which the NTK’s rate of change becomes negligible.

% -------------------------------------------------------
\subsection{Eigenvalue Drift Across Widths}

Let $\lambda_j(t)$ denote the eigenvalues of the empirical NTK at time $t$
(sorted in descending order).
For each width we track the top five eigenvalues:
\[
	\lambda_1(t),\;\lambda_2(t),\;\dots,\;\lambda_5(t).
\]

Empirically, we observe that the \emph{asymptotic} values of the top eigenvalues
are nearly identical across widths: regardless of the width, the leading eigenvalues
of $K(t)$ saturate toward essentially the same plateau.
What varies strongly with width is the \emph{timescale} on which this saturation occurs.

\begin{itemize}
	\item \textbf{Narrow networks} reach their asymptotic top eigenvalues quickly.
	      The leading eigenvalues stabilize early in training.
	\item \textbf{Wide networks} converge to the same eigenvalue plateau much more slowly,
	      even though the learning rate is identical for all widths.
\end{itemize}

This suggests that the empirical NTK of wide networks is less sensitive to
parameter updates for the same learning rate; the changes more gradually at large width.
Narrow networks, being further from the infinite-width NTK limit, exhibit
larger kernel changes per unit parameter movement, which leads to faster
spectral evolution.

Thus, the eigenvalue trajectories reveal a width-dependent \emph{spectral
	stabilization timescale}: small networks rapidly reshape their NTK spectrum,
while large networks evolve the spectrum slowly but eventually saturate at
similar eigenvalues.

\begin{figure}[H]
	\centering
	\includegraphics[width=.7\textwidth]{plots/eigenvalues/eigenvalues_w100.png}
	\caption{Top eigenvalues over training, width $W=100$.}
\end{figure}

\begin{figure}[H]
	\centering
	\includegraphics[width=.7\textwidth]{plots/eigenvalues/eigenvalues_w512.png}
	\caption{Top eigenvalues over training, width $W=512$.}
\end{figure}

\begin{figure}[H]
	\centering
	\includegraphics[width=.7\textwidth]{plots/eigenvalues/eigenvalues_w1024.png}
	\caption{Top eigenvalues over training, width $W=1024$.}
\end{figure}

\begin{figure}[H]
	\centering
	\includegraphics[width=.7\textwidth]{plots/eigenvalues/eigenvalues_w2048.png}
	\caption{Top eigenvalues over training, width $W=2048$.}
\end{figure}

\begin{figure}[H]
	\centering
	\includegraphics[width=.7\textwidth]{plots/eigenvalues/eigenvalues_w10000.png}
	\caption{Top eigenvalues over training, width $W=10000$.}
\end{figure}

% -------------------------------------------------------
\subsection{Loss Plateau and Kernel Drift}

We record the loss:
\[
	L(t) = \frac{1}{M}\sum_{i=1}^M
	\bigl(f_{\text{net}}(x_i,t) - y_i\bigr)^2.
\]

To test whether loss stabilization corresponds to NTK freezing,
we plot $L(t)$ alongside drift $\Delta_K(t)$ and slope $s(t)$.

The loss curve seems to have plateaued earlier than the time
$t_{\text{freeze}}$ at which the kernel drift slope becomes small, particularly for large widths.

\begin{figure}[H]
	\centering
	\includegraphics[width=.8\textwidth]{plots/loss_curves/loss_low_widths.png}
	\caption{Loss curves for low widths.}
\end{figure}

\begin{figure}[H]
	\centering
	\includegraphics[width=.8\textwidth]{plots/loss_curves/loss_mid_widths.png}
	\caption{Loss curves for mid-range widths.}
\end{figure}

\begin{figure}[H]
	\centering
	\includegraphics[width=.8\textwidth]{plots/loss_curves/loss_high_widths.png}
	\caption{Loss curves for high widths.}
\end{figure}

% =======================================================
\section{Kernel Regression Behavior}

\subsection{Kernel Regression Solution at Freeze Time}

Once a freeze time $t^*$ is detected, we construct the kernel regression solution
using the frozen empirical NTK:
\[
	\alpha(t^*) = \bigl(K_{\text{train,train}}(t^*) + \lambda I\bigr)^{-1}
	y_{\text{train}},
\]
and the corresponding evaluation-grid predictions:
\[
	f_{\mathrm{KR}}(\gamma; t^*)
	=
	K_{\text{eval,train}}(t^*)\,\alpha(t^*).
\]

This yields the function predicted by exact kernel regression at the freeze-time NTK.

\begin{figure}[H]
	\centering
	\includegraphics[width=.8\textwidth]{plots/kernel_regression/kernel_regression_w100_step.png}
	\caption{Kernel regression prediction using frozen NTK (width 100).}
\end{figure}

\begin{figure}[H]
	\centering
	\includegraphics[width=.8\textwidth]{plots/kernel_regression/kernel_regression_w512_step.png}
	\caption{Kernel regression prediction using frozen NTK (width 512).}
\end{figure}

\begin{figure}[H]
	\centering
	\includegraphics[width=.8\textwidth]{plots/kernel_regression/kernel_regression_w1024_step.png}
	\caption{Kernel regression prediction using frozen NTK (width 1024).}
\end{figure}

\begin{figure}[H]
	\centering
	\includegraphics[width=.8\textwidth]{plots/kernel_regression/kernel_regression_w2048_step.png}
	\caption{Kernel regression prediction using frozen NTK (width 2048).}
\end{figure}

% -------------------------------------------------------
\subsection{Comparison to Gradient-Descent Predictions}

In this section we examine how the network's predictions evolve \emph{after} the
NTK has effectively frozen.

For each width, we plot:
\begin{enumerate}
	\item the network prediction at the NTK freeze time $t^*$:
	      \[
		      f_{\text{net}}(\gamma, t^*),
	      \]
	\item the network prediction at the end of training:
	      \[
		      f_{\text{net}}(\gamma, t_{\text{final}}).
	      \]
\end{enumerate}

This comparison shows whether the network continues to change significantly in
function space after the NTK has stopped evolving.

If the NTK has truly entered the lazy regime, we expect the function to move
very little after $t^*$.

\begin{figure}[H]
	\centering
	\includegraphics[width=.8\textwidth]{plots/freeze_vs_final_spread/predictions_w100_freeze.png}
	\caption{Network predictions at freeze time vs.\ final time (width 100).}
\end{figure}

\begin{figure}[H]
	\centering
	\includegraphics[width=.8\textwidth]{plots/freeze_vs_final_spread/predictions_w512_freeze.png}
	\caption{Network predictions at freeze time vs.\ final time (width 512).}
\end{figure}

\begin{figure}[H]
	\centering
	\includegraphics[width=.8\textwidth]{plots/freeze_vs_final_spread/predictions_w1024_freeze.png}
	\caption{Network predictions at freeze time vs.\ final time (width 1024).}
\end{figure}

\begin{figure}[H]
	\centering
	\includegraphics[width=.8\textwidth]{plots/freeze_vs_final_spread/predictions_w2048_freeze.png}
	\caption{Network predictions at freeze time vs.\ final time (width 2048).}
\end{figure}

% =======================================================
\section{Summary}

\begin{itemize}
	\item \textbf{Freeze time vs.\ width:}
	      Wider networks freeze \emph{later}.
	      Narrow networks reach a small drift slope very early, while the widest
	      networks ($W=10000$) do not satisfy the freeze criterion even after
	      several million steps.

	\item \textbf{Eigenvalue stabilization timescales:}
	      All widths saturate to similar top eigenvalue values, but the rate of
	      saturation differs strongly:
	      narrow networks converge to their eigenvalue plateau quickly, whereas
	      wide networks saturate far more slowly.

	\item \textbf{Kernel regression vs.\ target accuracy:}
	      Kernel regression behaves differently across widths:
	      \begin{itemize}
		      \item For \textbf{narrow networks}, the kernel-regression
		            prediction computed at the freeze time is \emph{closer to
			            the target function} than the final trained network.
		            In this regime, the finite-width network drifts away from the
		            KR solution as training progresses.
		      \item For \textbf{wide networks}, the \emph{final trained network}
		            is closer to the target than the kernel regression solution.
		            Kernel regression underperforms relative to gradient descent
		            at large width.
	      \end{itemize}
\end{itemize}

\begin{figure}[H]
	\centering
	\includegraphics[width=.85\textwidth]{plots/error_vs_width_combined.png}
	\caption{
		Kernel regression error and final network error versus width.
		Narrow networks show lower kernel-regression error than final-network error,
		while for wide networks the situation reverses.
	}
\end{figure}

% =======================================================
\bibliographystyle{apalike}
\bibliography{references}

\end{document}
