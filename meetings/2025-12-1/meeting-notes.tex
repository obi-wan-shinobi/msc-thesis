\documentclass[a4paper,11pt]{article}
\usepackage[utf8]{inputenc}
\usepackage[T1]{fontenc}
\usepackage{lmodern}
\usepackage{amsmath,amssymb}
\usepackage{graphicx}
\usepackage{caption}
\usepackage{geometry}
\geometry{margin=1in}
\usepackage{hyperref}
\usepackage[round,sort&compress]{natbib}
\hypersetup{colorlinks=true,linkcolor=black,citecolor=black,urlcolor=blue}

\usepackage{float}
\usepackage[section]{placeins}

\title{\textbf{EXP003: Mode-wise Decay of Training Residuals}\\[2mm]
\large NTK Eigenmodes and Fourier Mixture Components}
\author{Shreyas Kalvankar}
\date{December 2025}

\begin{document}
\maketitle

\section*{Purpose of This Analysis}

This brief note presents two diagnostics that track how learning progresses
along different function-space directions:

\begin{enumerate}
	\item \textbf{NTK eigenmode decay:} projection of the training residual
	      onto the top 5 eigenvectors of the empirical NTK at each training step.

	\item \textbf{Fourier mixture component decay:}
	      projection of the residual onto the seven components of the target
	      Fourier mixture:
	      \[
		      f^*(\gamma) = \sum_{j=1}^7 a_j \sin(K_j\gamma + \phi_j),
		      \qquad K_j \in \{2,4,7,11,16,23,32\}.
	      \]
\end{enumerate}

These plots reveal:
\begin{itemize}
	\item which eigenmodes are reduced earliest during training;
	\item which Fourier harmonics of the target are learned early vs.\ late;
	\item how these behaviours depend on network width.
\end{itemize}

% =======================================================
\section{Residual Projections onto NTK Eigenmodes}

For each snapshot step $t$, the empirical NTK on the training set is decomposed
as $K_{tt}(t) = U(t)\Lambda(t)U(t)^\top$.
We track the magnitude of the projection of the residual $r_t$
onto the top five eigenvectors.

A dashed line indicates the NTK freeze time for each width.

\begin{figure}[H]
	\centering
	\includegraphics[width=.8\textwidth]{plots/mode_decay/mode_decay_w100.png}
	\caption{NTK eigenmode decay, width 100.}
\end{figure}

\begin{figure}[H]
	\centering
	\includegraphics[width=.8\textwidth]{plots/mode_decay/mode_decay_w512.png}
	\caption{NTK eigenmode decay, width 512.}
\end{figure}

\begin{figure}[H]
	\centering
	\includegraphics[width=.8\textwidth]{plots/mode_decay/mode_decay_w1024.png}
	\caption{NTK eigenmode decay, width 1024.}
\end{figure}

\begin{figure}[H]
	\centering
	\includegraphics[width=.8\textwidth]{plots/mode_decay/mode_decay_w2048.png}
	\caption{NTK eigenmode decay, width 2048.}
\end{figure}

\begin{figure}[H]
	\centering
	\includegraphics[width=.8\textwidth]{plots/mode_decay/mode_decay_w10000.png}
	\caption{NTK eigenmode decay, width 10000.}
\end{figure}

% =======================================================
\section{Residual Projections onto Fourier Mixture Components}

Let the seven basis vectors be:
\[
	b_j(\gamma) = \sin(K_j \gamma + \phi_j),
	\qquad K_j \in \{2,4,7,11,16,23,32\}.
\]

For each step, we track the magnitude of the residual projected onto these
components:
\[
	c_j(t)
	= \frac{\langle r_t,\, b_j\rangle}{\langle b_j, b_j\rangle}.
\]

This shows how rapidly each harmonic of the target is learned.

\begin{figure}[H]
	\centering
	\includegraphics[width=.8\textwidth]{plots/fourier_decay/fourier_decay_w100.png}
	\caption{Fourier mixture component decay, width 100.}
\end{figure}

\begin{figure}[H]
	\centering
	\includegraphics[width=.8\textwidth]{plots/fourier_decay/fourier_decay_w512.png}
	\caption{Fourier mixture component decay, width 512.}
\end{figure}

\begin{figure}[H]
	\centering
	\includegraphics[width=.8\textwidth]{plots/fourier_decay/fourier_decay_w1024.png}
	\caption{Fourier mixture component decay, width 1024.}
\end{figure}

\begin{figure}[H]
	\centering
	\includegraphics[width=.8\textwidth]{plots/fourier_decay/fourier_decay_w2048.png}
	\caption{Fourier mixture component decay, width 2048.}
\end{figure}

\begin{figure}[H]
	\centering
	\includegraphics[width=.8\textwidth]{plots/fourier_decay/fourier_decay_w10000.png}
	\caption{Fourier mixture component decay, width 10000.}
\end{figure}

% =======================================================
\section{Key Observations}

\begin{itemize}
	\item \textbf{Eigenmodes:}
	      Across all widths, the top-5 mode amplitudes show non-monotonic behaviour
	      (dips, rebounds, long drifts), and do not collapse after the NTK freeze
	      point. This indicates that feature-learning–type dynamics continue even
	      when the empirical NTK appears nearly frozen in Frobenius norm.
	\item Increasing width smooths the trajectories but does not eliminate this behaviour:
	      wider networks show less noise and slower drift.
	      This suggests that NTK freezing (as measured by drift) is not sufficient to imply lazy-training behaviour at
	      finite width.

	\item \textbf{Fourier harmonics:}
	      Low frequencies ($k=2,4$) decay early, while higher frequencies
	      ($k\ge 16$) decay much more slowly, confirming a strong spectral bias.
\end{itemize}

\end{document}
