\documentclass[11pt,a4paper]{article}

% --- packages ---
\usepackage[utf8]{inputenc}
\usepackage[T1]{fontenc}
\usepackage{lmodern}
\usepackage{microtype}
\usepackage{amsmath,amssymb,amsthm,mathtools,bm}
\usepackage{hyperref}
\usepackage[round,authoryear]{natbib}
\usepackage{geometry}

\geometry{margin=1in}

% --- title info ---
\title{Thesis Proposal \\[0.5em]
Training and Generalization of Overparametrized Neural Networks}
\author{Shreyas Kalvankar}
\date{\today}

\begin{document}
\maketitle

\section{Introduction}
Overparametrized neural networks (models with more parameters than training
samples) are capable of perfectly interpolating data, yet they often generalize
surprisingly well. The \emph{Neural Tangent Kernel} (NTK) framework
\citep{jacot2018ntk,lee2020wide} provides a rigorous description of infinitely
wide networks: in this regime, training is equivalent to kernel regression with
a fixed kernel, so optimization and generalization can be analyzed with
classical kernel methods. However, this ``lazy training'' limit rules out
feature learning, since the kernel does not evolve during training.

Recent results show that in realistic finite-width and finite-depth networks,
the NTK is random at initialization and evolves non-trivially during training
\citep{haninNica2019finite}. This suggests the existence of a ``weak feature
learning regime,'' where kernel evolution remains small but non-negligible,
interpolating between lazy training and full representation learning.

This thesis will develop an analysis of the training dynamics under an evolving
NTK, quantify how kernel drift scales with the depth($d$)/width($n$) ratio $\beta=d/n$,
and interpret the induced sequence of RKHSs $(\mathcal H_t)_{t\geq0}$ to
understand the regimes where weak feature learning arises and how it affects
generalization.

\section{Research Question}
\textbf{Main research question:}
\begin{quote}
	How can we characterize the evolution of the NTK in finite-width and finite-depth neural networks, and what does this imply for effective function spaces and generalization?
\end{quote}

% \subsection*{Sub-questions (current version)}
% \begin{enumerate}
% 	\item How can NTK evolution be expressed analytically, for example via perturbation of the lazy kernel $K_0$?
% 	\item How does kernel drift affect the gradient flow equation $\dot f_t = -K_t(f_t-y)$ compared to the lazy prediction?
% 	\item Can we bound or characterize the regimes (lazy, weak feature learning, unstable) in terms of $\beta=d/n$?
% 	\item How can NTK evolution be interpreted as movement between function spaces (e.g.\ RKHSs)?
% 	      % \item How do higher-order expansions \citep{bai2020linearizationquadratichigherorderapproximation} relate to kernel evolution corrections?
% \end{enumerate}

\subsection*{Sub-questions}
\begin{enumerate}
	\item How can the NTK evolution be written as $K_t = K_0 + \Delta K_t$, and how does this perturbation affect the gradient flow dynamics relative to $K_0$?
	\item How does the size of $\Delta K_t$ scale with $\beta = d/n$? Can we identify thresholds for when $\Delta K_t$ significantly alters training dynamics?
	\item How can we describe the evolving RKHSs $(\mathcal H_t)$ induced by $K_t$? Do these spaces enlarge compared to $\mathcal H_0$ and allow approximation of functions outside $\mathcal H_0$?
	\item How does movement between RKHSs affect the inductive bias, e.g.\ via eigenvalue decay of $K_t$ and the minimal-norm interpolant in $\mathcal H_t$?
\end{enumerate}

\section{Background}

\subsection{Overparametrized Neural Networks}
Overparametrized models often interpolate the training data yet generalize
well. This paradoxical behavior connects to the implicit bias of gradient-based
optimization, whereby gradient descent converges to specific low-complexity
solutions among many possible interpolants
\citep{soudry2024implicit,gunasekar2018implicit}. Phenomena such as double
descent and benign overfitting underscore the need for new analytical
frameworks.

\subsection{Neural Tangent Kernel}
The NTK provides one such framework: by linearizing a network at
initialization, one defines a kernel that describes parameter coupling during
training \citep{jacot2018ntk}. In the infinite-width limit, this kernel is
deterministic and fixed, reducing training to kernel regression
\citep{lee2020wide}. While elegant, this ``lazy training'' regime excludes
feature learning, since the feature map is frozen at initialization.

\subsection{Finite Depth/Width Corrections}
\citet{haninNica2019finite} show that when depth and width co-scale, the NTK is
stochastic and evolves over training. They identify $\beta = d/n$ as a key
scaling parameter: the NTK variance grows like $\exp(c\beta)$, and even the
first SGD update changes the kernel. This suggests a weak feature learning
regime $0<\beta\ll 1$ where kernel drift is non-negligible but stable.

\subsection{Function Spaces and RKHS Perspective}
Every positive definite kernel $K$ defines a Reproducing Kernel Hilbert Space
(RKHS) $\mathcal H$, consisting of functions with an inner product structure
induced by $K$. In the NTK framework, the infinite-width limit corresponds to
kernel regression in the RKHS $\mathcal H_0$ induced by the initialization
kernel \citep{jacot2018ntk,bartolucci2023rkbs}. This space encodes the inductive bias of lazy
training: solutions are minimum-norm interpolants in $\mathcal H_0$. When the
kernel evolves during training, the effective hypothesis space can be viewed as
a sequence of RKHSs $(\mathcal H_t)_{t\geq0}$. Studying how these spaces change
provides an interpretation of weak feature learning.

% \subsection{Beyond NTK}
% \citet{bai2020linearizationquadratichigherorderapproximation} propose higher-order expansions of network dynamics, capturing quadratic and beyond-linear terms in the training trajectory.
% These corrections highlight alternative analytical approaches to finite-width behavior and motivate comparison with kernel-evolution perspectives.

\section{Research Gap}
Current theory establishes that the NTK is fixed at infinite width and evolves when depth and width co-scale.
However, several gaps remain:
\begin{itemize}
	\item No formal perturbation analysis connecting kernel drift to deviations in training dynamics.
	\item Limited characterization of the boundary between lazy and weak feature learning regimes in terms of $\beta$.
	\item Lack of interpretation of evolving NTKs as a path of RKHSs.
	      % \item No unifying comparison between finite-width NTK evolution and higher-order Taylor expansions.
\end{itemize}

\section{Scope}
The work will be analytical in nature, focusing on:
\begin{itemize}
	\item Perturbation expansions of the training dynamics under evolving $K_t$.
	\item Bounds on kernel drift as a function of depth/width scaling.
	\item Analysis of the RKHS sequence $(\mathcal H_t)_{t\geq0}$ induced by $K_t$.
	      % \item Conceptual comparison between NTK corrections and higher-order expansions.
\end{itemize}

\section{Methodology}
\subsection{Perturbation Analysis of Kernel Dynamics}
Write $K_t = K_0 + \Delta K_t$, with $K_0$ the lazy NTK.
Substitute into the gradient flow ODE and develop a perturbation expansion.
% \[
% 	f_t \approx y + e^{-K_0t}(f_0-y) + \int_0^t e^{-K_0(t-s)} \Delta K_s (f_s-y)\, ds.
% \]
Analyze conditions on $\Delta K_t$ (e.g.\ scaling in $\beta$) under which corrections are small or large.
%
% \medskip
% \textbf{Note.} This section still needs brainstorming.

\subsection{Scaling Regimes}
The aim here is to understand how the depth/width ratio $\beta = d/n$ marks
transitions between lazy, weak feature learning, and unstable regimes. This
involves considering how measures of kernel drift depend on $\beta$, in light
of theoretical results such as Hanin--Nica’s exponential scaling.
%
% \medskip
% \textbf{Note.} This section still needs brainstorming.

\subsection{Functional Analysis Perspective}
We view training as generating a family of RKHSs $\{\mathcal H_t\}_{t\ge 0}$,
where each $\mathcal H_t$ is induced by the kernel $K_t$. This provides a
function-space lens on kernel evolution where changes in $K_t$ correspond to changes
in the associated norm and hypothesis space. The question is then can we use this
to understand generalization?
%
% \medskip
% \textbf{Note.} This section still needs brainstorming.

\section{Expected Outcomes}
\begin{itemize}
	\item Perturbation-theoretic description of NTK evolution and its effect on
	      training dynamics.
	\item Theoretical characterization of scaling regimes in terms of $\beta$
	      (lazy, weak feature learning, unstable).
	\item Functional analysis interpretation of evolving NTKs as a family of RKHSs.
	      % \item Conceptual comparison between kernel-evolution corrections and higher-order Taylor expansions.
\end{itemize}

\section{Tentative Timeline (partial)}\label{sec:Tentative Timeline} % (fold)
\begin{itemize}
	\item \textbf{Weeks 1--2} \\
	      Refine understanding of NTK theory and finite-width corrections.
	      Implement a small-scale JAX setup to reproduce key diagnostics from
	      \citet{haninNica2019finite}
	      \begin{enumerate}
		      \item Verify the explosion of the normalized second moment of $K_N(x,x)$ at initialization,
		            scaling like $\exp(5\beta)$ with $\beta = d/n$.
		      \item Measure the expected first-step kernel update $\Delta K$, confirming its growth with $\beta$
		            as predicted by their theory.
	      \end{enumerate}

	\item \textbf{Weeks 3--6} \\
	      Develop the perturbative description $K_t = K_0 + \Delta K_t$.
	      Derive preliminary bounds or conditions on $\Delta K_t$. Set up some
	      experiments to test the bounds.

	\item \textbf{Weeks 7--10} \\
	      Analyze how $\Delta K_t$ depends on $\beta = d/n$ using existing
	      results (e.g., Hanin--Nica). Try to characterize regime boundaries
	      (lazy, weak feature learning, unstable).

	\item \textbf{Weeks 11--14} \\
	      Frame $K_t$ as inducing a family of RKHSs $(\mathcal H_t)_{t\ge 0}$.
	      Explore preliminary results on how $\mathcal H_t$ compares to $\mathcal H_0$ in terms of approximation or inductive bias.

\end{itemize}
% section Tentative Timeline (end)

\bibliographystyle{plainnat}
\bibliography{bibliography}

\end{document}
