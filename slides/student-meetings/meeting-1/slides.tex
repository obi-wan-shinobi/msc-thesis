\documentclass[10pt,aspectratio=169]{beamer}

% --- pdflatex-safe font & micro-typography ---
\usepackage[T1]{fontenc}
\usepackage[utf8]{inputenc}
\usepackage{lmodern}
\usepackage{microtype}

% --- math & symbols ---
\usepackage{amsmath,amssymb,mathtools,bm}

% --- hyperlinks (Beamer already loads hyperref; this is safe) ---
\hypersetup{hidelinks}

% --- theme ---
\usetheme[numbering=fraction]{metropolis}

\usepackage{pgfpages} % for notes layouts

% --- bibliography ---
\usepackage[round,authoryear]{natbib}

% --- Choose ONE of these toggles when compiling ---
% \setbeameroption{hide notes}                        % audience version (default)
% \setbeameroption{show notes}                        % notes printed under each slide
% \setbeameroption{show only notes}                   % notes-only PDF (for printing)
% \setbeameroption{show notes on second screen=right} % slide + notes side-by-side

% --- shortcuts ---
\newcommand{\E}{\mathbb{E}}
\newcommand{\R}{\mathbb{R}}
\newcommand{\btheta}{\boldsymbol{\theta}}
\newcommand{\bx}{\boldsymbol{x}}

% --- title info ---
\title{Infinite Width, NTK, and Feature Learning}
\subtitle{Student Meeting 1}
\author{Shreyas Kalvankar}
\institute{TU Delft}
\date{\today}

\begin{document}

\maketitle

% -----------------------
\begin{frame}{Agenda}
	\textbf{Goal of this talk}
	\begin{itemize}
		\item Introduce the infinite-width lens, NTK, and the lazy training regime via a simple one-layer example.
		\item Situate these ideas within related work: baseline results at infinite width and viewpoints that go beyond linearization.
	\end{itemize}
	\medskip
	\textbf{What I would like from you}
	\begin{itemize}
		\item Conceptual clarifications and critiques of how I’m connecting the papers.
		\item Suggestions for key references I might be missing.
		\item Pointers to alternative frameworks worth comparing (mean-field, functional-analytic, optimization bias).
	\end{itemize}

	\note{
		Emphasize this is a literature discussion: I’m laying out how different results fit together,
		not presenting a concrete plan. Invite refs and corrections rather than action items.
	}
\end{frame}

% -----------------------
\begin{frame}{Why theory?}
	\begin{itemize}
		\item Larger models keep improving, but we don’t fully know \emph{why}.
		      \pause
		\item \textbf{A solvable starting point:} infinite width gives a clean
		      baseline; modern models are large-but-finite, deep, and do feature
		      learning.
		      \pause
		\item \textbf{What can theory help in:} explanations of
		      convergence/generalization, signals for when features move, and
		      guidance on design choices (init, learning rate, normalization).
	\end{itemize}

	\note{
		Scaling helps, but the mechanisms are unclear. Infinite-width theory is
		a useful baseline, not the endpoint—real models are large-but-finite
		and learn features. We can study these settings to see what theory
		already explains about convergence and when feature learning kicks in.
	}
\end{frame}

% -----------------------

% -----------------------
\begin{frame}{Motivation}
	\begin{itemize}
		\item \textbf{Infinite width:} clean, analyzable baseline; randomness averages to a deterministic kernel.
		      \pause
		\item \textbf{NTK:} first-order (linear) view of training; clear convergence intuition via a fixed kernel.
		      \pause
		\item \textbf{Beyond the linear NTK picture:} when does it stop being accurate?
		      \begin{itemize}
			      \item Does the \emph{kernel} change during training?
			            \pause
			      \item Do the model’s internal features move?
			            \pause
			      \item Do training curves deviate from the NTK baseline prediction?
			            \pause
			      \item Possible causes: higher-order effects, useful parameter
			            re-organization that push the model out of the lazy regime.
		      \end{itemize}
	\end{itemize}
\end{frame}

% -----------------------
\begin{frame}{Setup}
	\textbf{Supervised learning:} data $\{(\bx_i,y_i)\}_{i=1}^n$, $\bx_i\in\R^d, y_i\in\R$.

	\pause
	Neural network $f:\R^d\times\Theta\to\R$ with parameters $\btheta\in \Theta \subset \R^p$.
	\pause

	Trained by (continuous-time) gradient flow on squared loss:
	\[
		L(\btheta)=\tfrac{1}{2}\sum_{i=1}^n (f(\bx_i;\btheta)-y_i)^2,
		\qquad
		\pause
		\frac{d}{dt}\btheta_t=-\nabla_{\btheta} L(\btheta_t).
	\]
\end{frame}

% -----------------------
\begin{frame}{Linearization at Initialization}
	First-order Taylor around $\btheta_0$:
	\[
		f(\bx;\btheta)\approx f(\bx;\btheta_0)+\underbrace{\nabla_{\btheta} f(\bx;\btheta_0)^{\top}}_{:=~\phi(\bx)^{\top}}(\btheta-\btheta_0).
	\]
	\pause
	This induces the \textbf{Neural Tangent Kernel (NTK)} at $\btheta_0$:
	\[
		K(\bx,\bx')=\phi(\bx)^{\top}\phi(\bx')=\nabla_{\btheta} f(\bx;\btheta_0)^{\top}\nabla_{\btheta} f(\bx';\btheta_0).
	\]

	\vspace{0.4em}
	\footnotesize \emph{Ref:} \citep{jacot2018ntk}\normalsize
\end{frame}

% -----------------------
\begin{frame}{Illustrative Model (One Hidden Layer, No Biases)}
	Consider a \textit{m}-width neural network

	\[
		f(\bx)=\frac{1}{\sqrt{m}}\sum_{r=1}^m a_r\, \sigma(\mathbf{w}_r^{\top} \bx),\quad \pause
		a_r\sim\mathcal N(0,\sigma_a^2),\;
		\mathbf{w}_r\sim\mathcal N\!\Big(0,\tfrac{\sigma_w^2}{d}I\Big).
	\]

	\pause
	Gradients:
	\[
		\nabla_{a_r} f(\bx)=\frac{1}{\sqrt{m}}\sigma(\mathbf{w}_r^{\top} \bx),\quad
		\pause
		\nabla_{\mathbf{w}_r} f(\bx)=\frac{1}{\sqrt{m}}\,a_r\,\sigma'(\mathbf{w}_r^{\top} \bx)\,\bx.
	\]
	\pause
	Finite-width NTK:
	\[
		\boxed{
		K_m(\bx,\bx')
		=\frac{1}{m}\sum_{r=1}^m \sigma(\mathbf{w}_r^{\top} \bx)\,\sigma(\mathbf{w}_r^{\top} \bx')
		+\frac{1}{m}\sum_{r=1}^m a_r^{2}\,\sigma'(\mathbf{w}_r^{\top} \bx)\,\sigma'(\mathbf{w}_r^{\top} \bx')\,\bx^{\top}\bx'.
		}
	\]
\end{frame}

% -----------------------
\begin{frame}{Infinite-Width NTK}
	The two sums are empirical averages of i.i.d.\ terms. Since $a_r$ and $w_r$ are independent with finite moments,
	the (strong) law of large numbers gives, almost surely, as width $m\to\infty$:
	\pause
	\[
		\frac{1}{m}\sum_{r=1}^m \sigma(w_r^\top x)\,\sigma(w_r^\top x')
		\;\longrightarrow\;
		\mathbb E_{w}\big[\sigma(w^\top x)\,\sigma(w^\top x')\big],
	\]
	\pause
	\[
		\frac{1}{m}\sum_{r=1}^m a_r^2\,\sigma'(w_r^\top x)\,\sigma'(w_r^\top x')
		\;\longrightarrow\;
		\sigma_a^2\,\mathbb E_{w}\big[\sigma'(w^\top x)\,\sigma'(w^\top x')\big].
	\]
	\pause
	Thus, in the infinite-width limit, the empirical NTK converges almost surely to a deterministic kernel
	\[
		\boxed{
			K_{\infty}(\bx,\bx')=
			\E_{\mathbf{w}}\!\big[\sigma(\mathbf{w}^{\top}\bx)\sigma(\mathbf{w}^{\top}\bx')\big]
			+\sigma_a^2\,\bx^{\top}\bx'\,\E_{\mathbf{w}}\!\big[\sigma'(\mathbf{w}^{\top}\bx)\sigma'(\mathbf{w}^{\top}\bx')\big].
		}
	\]
	\pause
	Key point: in the wide limit, $K_t\approx K_0$ remains \emph{essentially constant} during training (lazy regime).
	\vspace{0.4em}
	\footnotesize \emph{Refs:} \citep{neal1996priors,lee2019wide}\normalsize

	\note{
		Explain the logic slowly:
		\begin{itemize}
			\item Each of the two sums is just an average of i.i.d.\ terms (from random init).
			\item By the law of large numbers, as $m\to\infty$ they converge to their expectations.
			\item That’s why the NTK at infinite width is \emph{deterministic} — randomness washes out.
			\item In this regime, $K_t\approx K_0$: the kernel stays constant during training.
			\item This is the essence of the \textbf{lazy training regime}: no feature movement, training = kernel regression.
		\end{itemize}
		Emphasize: this is a clean baseline, but also a limitation — finite networks won’t behave this way exactly.
	}
\end{frame}

% -----------------------
\begin{frame}{Training dynamics: mini-derivation}
	\textbf{Gradient flow \& chain rule:} Let $f_t(x_i):=f(x_i;\theta_t)$. Then
	\[
		\frac{d}{dt} f_t(x_i)
		= \nabla_\theta f(x_i;\theta_t)^{\top}\,\dot{\theta}_t
		= -\,\nabla_\theta f(x_i;\theta_t)^{\top}\,\nabla_\theta L(\theta_t).
	\]
	\pause
	\textbf{Loss gradient (squared loss):}
	\[
		L(\theta)=\tfrac{1}{2}\sum_{j=1}^n \big(f(x_j;\theta)-y_j\big)^2,
		\qquad
		\nabla_\theta L(\theta_t)
		= \sum_{j=1}^n \big(f_t(x_j)-y_j\big)\,\nabla_\theta f(x_j;\theta_t).
	\]
\end{frame}

\begin{frame}{Training dynamics: constant kernel}
	\textbf{Substitute and define the (time–dependent) NTK.}
	\[
		\frac{d}{dt} f_t(x_i)
		= -\sum_{j=1}^n \underbrace{\nabla_\theta f(x_i;\theta_t)^{\top}\nabla_\theta f(x_j;\theta_t)}_{=:~K_t(x_i,x_j)}
		\big(f_t(x_j)-y_j\big).
	\]
	\pause
	\textbf{Vectorized form:}
	\[
		\boxed{~\dot f_t \;=\; -\,K_t\,(f_t-y).~}
	\]
	\pause
	\textbf{Constant-kernel (NTK) regime.} If $K_t \approx K_0 \equiv K$ (infinite width / lazy),
	\[
		\dot f_t = -K(f_t-y)
		\quad\Rightarrow\quad
		\boxed{~f_t = y + e^{-Kt}\,(f_0-y).~}
	\]

	\vspace{0.4em}
	\footnotesize \emph{Refs:} \citep{jacot2018ntk,lee2019wide}\normalsize

	\note{
		Read: gradient flow + chain rule $\Rightarrow$ $\dot f_t=-K_t(f_t-y)$. In the constant-kernel regime $K_t\approx K$, we get linear kernel flow with closed-form solution.
	}
\end{frame}

% -----------------------
\begin{frame}{Lazy Training}
	\textbf{Lazy regime:} parameter drift is small, $\|\btheta_t-\btheta_0\|\ll \|\btheta_0\|$.
	\begin{itemize}
		\item Features $\phi(\bx)$ and kernel $K$ stay (approximately) constant.
		\item Training reduces to kernel regression with fixed $K$.
	\end{itemize}
	\textbf{Limitation:} suppresses \emph{feature learning} (representation change).
	\medskip

	\textbf{Aim:} quantify \emph{when} lazy holds/breaks and \emph{how} to model beyond it.
\end{frame}
% -----------------------
\begin{frame}{Diagnosing the transition to feature learning}
	\textbf{Some signals to consider}
	\begin{itemize}
		\item \textbf{Kernel drift:} Does the NTK matrix $K_t$ change during training?
		      Compare $K_t$ to $K_0$ on the (same) data. Bigger change $\Rightarrow$ more feature learning.
		      \pause
		\item \textbf{Feature drift:} Do the tangent features
		      $\phi_t(x)=\nabla_\theta f(x;\theta_t)$ move on a small fixed set $\mathcal S$?
		      Track the average change from $t=0$.
	\end{itemize}

	\pause
	\medskip
	\textbf{How to (often) push out of lazy}
	\begin{itemize}
		\item larger learning rate \hfill \item smaller width \hfill \item more depth / biases / normalization
	\end{itemize}

	\note{
		Quick metrics (optional): kernel drift $\approx \|K_t-K_0\|/\|K_0\|$ (Frobenius or operator norm);
		feature drift on $\mathcal S$: average relative change or 1–cosine between $\phi_t(x)$ and $\phi_0(x)$.
		Rising drift over time indicates leaving the lazy (fixed-kernel) regime.
	}
\end{frame}
% -----------------------
\begin{frame}{Beyond linearization I: quadratic / higher-order}
	\[
		f(\bx;\btheta) \approx f(\bx;\btheta_0)
		+ \phi(\bx)^{\top}\Delta\btheta
		+ \tfrac12\,\Delta\btheta^{\top} H_f(\bx)\,\Delta\btheta,
		\qquad \Delta\btheta=\btheta-\btheta_0.
	\]
	\begin{itemize}
		\item \textbf{Mechanism (\citet{bai2020beyond}):} construct regimes where the linear term is suppressed so the \emph{quadratic} term governs the dynamics; extendable to $k\!>\!2$ (“higher-order NTKs”).
		\item \textbf{Findings:} with the linear term suppressed, progress
		      comes from \emph{feature changes}; this adaptive
		      regime is easy to optimize and can beat NTK in sample use on simple
		      tasks.
	\end{itemize}
	\vspace{0.3em}
	\footnotesize\emph{See: “Beyond Linearization: On Quadratic and Higher-Order Approximation of Wide Neural Networks.”}\normalsize
\end{frame}

% -----------------------
\begin{frame}{Beyond linearization II: adaptive / time-varying kernels}
	\begin{itemize}
		\item \citet{zhang2024beyondntk} replace the fixed NTK with a \emph{time-varying} kernel $K_t$ that evolves during training (“kernel drift”).
		\item Features adapt during training and increasingly align with label-relevant directions (growing alignment).
		\item They provide a prototype of an over-parameterized Gaussian sequence model to analyze feature learning beyond the NTK picture.
	\end{itemize}
	\vspace{0.3em}
	\footnotesize\emph{See: “Towards a Statistical Understanding of Neural Networks: Beyond the NTK Theories.”}\normalsize
\end{frame}

% -----------------------
\begin{frame}{Beyond NTK III: finite depth/width corrections}
	\begin{itemize}
		\item \citet{haninNica2019finite} study networks where depth $d$ and width $n$ grow together.
		\item Key finding: the NTK is \textbf{not deterministic} at init — its variance scales
		      roughly $\exp(c\,d/n)$.
		\item Even the first SGD step can change $K_t$ significantly $\;\Rightarrow\;$ kernels evolve, enabling
		      \emph{weak feature learning}.
		\item Proposed \textbf{weak feature learning regime:} $0<d/n\ll 1$ where training is stable but $K_t$ still moves.
	\end{itemize}
	\vspace{0.3em}
	\footnotesize\emph{See: Hanin \& Nica, “Finite Depth and Width Corrections to the NTK.”}\normalsize
\end{frame}
% -----------------------

% % -----------------------
% \begin{frame}{How This Extends to General Networks (Project Scope)}
% 	We keep derivations simple for slides, but the program targets:
% 	\begin{enumerate}
% 		\item \textbf{General architectures} (biases, depth, common activations).
% 		\item \textbf{Initialization/scale choices} ensuring well-posed preactivations.
% 		\item \textbf{Continuous-time training} (gradient flow) for clarity.
% 	\end{enumerate}
% 	\textbf{Plan:} Use first-order NTK to set the baseline, then introduce controlled
% 	second-order corrections that remain tractable at scale (statistics of $H_f$ under initialization).
% \end{frame}

% -----------------------

% -----------------------
\begin{frame}{Summary \& discussion}
	\begin{itemize}
		\item Infinite width as a clean baseline; NTK via linearization at initialization.
		\item Constant–kernel training dynamics: $\dot f_t=-K(f_t-y)$ with solution $f_t=y+e^{-Kt}(f_0-y)$.
		\item Lazy training $\Rightarrow$ features (and $K$) stay essentially fixed.
		\item How to spot leaving lazy: \emph{kernel drift} ($K_t\neq K_0$) and \emph{feature drift} on a probe set.
		\item Beyond NTK in the literature:
		      \begin{itemize}
			      \item \emph{Quadratic / higher-order} near init \citep{bai2020beyond}.
			      \item \emph{Adaptive / time-varying kernels} and alignment \citep{zhang2024beyondntk}.
			      \item \emph{Finite depth/width corrections:} NTK variance grows with $d/n$; even early updates can move $K_t$ (weak feature learning) \citep{haninNica2019finite}.
		      \end{itemize}
	\end{itemize}
\end{frame}
% -----------------------
\begin{frame}{References}
	\bibliographystyle{abbrvnat}  % or plainnat
	\bibliography{refs}
\end{frame}
% -----------------------
\end{document}
